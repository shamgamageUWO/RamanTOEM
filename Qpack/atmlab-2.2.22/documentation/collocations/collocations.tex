% For now, this is a seperate document.
% However, if more components of Atmlab grow documentation in the future, this
% may become part of a larger body of Atmlab documentation.
%\documentclass[a4paper,10pt]{memoir}
\documentclass[a4paper,10pt]{article}
\usepackage{standalone}
\usepackage[T1]{fontenc}
\usepackage{lmodern}
\usepackage{extsizes} % redundant with memoir
\usepackage{natbib}
\setcitestyle{authoryear,round,semicolon}
\usepackage{paralist}
\usepackage[colorlinks=true,allcolors=black]{hyperref}
\usepackage[style=index, acronym]{glossaries}
\usepackage{microtype}
\usepackage{amsmath}
\usepackage{amssymb}
\usepackage{textcomp}
\usepackage{siunitx}
\sisetup{seperr,repeatunits=false}
\usepackage{booktabs}
\usepackage{tikz}
\usepackage{pgfplots}
\pgfplotsset{compat=1.5}
%\usetikzlibrary{arrows,positioning,backgrounds,fit,shapes,calc}
%\usepgfplotslibrary{units}
\hbadness=10000 % don't bother be about underfull hbox
\vbadness=10000 % idem dito

\usepackage[utf8]{inputenc}
\usepackage[english]{babel}
\usepackage[english]{isodate}
\usepackage{url}
\usepackage{pgf-umlcd}
\usepackage{listings}
\lstset{language=Matlab}
\makeglossaries
% For use with \usepackage[acronym]{glossaries}.
%
% \input this file after you have defined your own acronyms (but why not
% add your own acronyms here?). It will define any acronyms that you have
% yet defined yourself (acronyms will not be over-written). The acronym
% key will be equal to the actual acronym, e.g. \gls{ARTS}.
%
% You don't need to include an actual glossary to use those acronyms; the
% package takes care of the first usage \gls{ABC} being expanded. See
% 'texdoc glossaries' for more info.
%
% If you do include a glossary (table of acronyms), only those acronyms
% that are actually used will be included.
%
% $Id: acronyms.tex 7678 2012-04-04 16:30:11Z gerrit $

\newcommand{\makeacroif}[2]{%
\ifglsentryexists{#1}{}{%
 \newacronym{#1}{#1}{#2}}}

\newcommand{\makepluralacroif}[3]{%
\ifglsentryexists{#1}{}{%
 \newacronym[plural=#2]{#1}{#1}{#3}}}

\makeacroif{IPCC}{Intergovernmental Panel on Climate Change}
\makeacroif{AR4}{4th Assessment Report}
\makeacroif{NOAA}{National Oceanic and Atmospheric Administration}
\makeacroif{TRMM}{Tropical Rainfall Measuring Mission}
\makeacroif{CALIPSO}{Cloud-Aerosol Lidar and Infrared Pathfinder Satellite Observation}
\makeacroif{PARASOL}{Polarization and Anisotropy of Reflectances
                              for Atmospheric Sciences Coupled with
                              Observations from a Lidar}
\makeacroif{OCO}{Orbital Carbon Observatory}
\makeacroif{ESMR}{Electrically Scanning Microwave Radiometer}
\makeacroif{DMSP}{Defense Meteorological Satellite Program}
\makeacroif{SSM/T}{Special Sensor Microwave/Temperature}
\makeacroif{SEVIRI}{Spinning Enhanced Visible Infra-Red Imager}
\makeacroif{VHRR}{Very High Resolution Radiometer}
\makeacroif{UV}{Ultra-Violet}
\makeacroif{IR}{Infra-Red}
\makeacroif{MW}{microwave}
\makeacroif{FRAC}{Full Resolution Area Coverage}
\makeacroif{GAC}{Global Area Coverage}
\makeacroif{IWSSM}{International Workshop on Space-based Snowfall Measurement}
\makeacroif{DDA}{Discrete Dipole Approximation}
\makeacroif{PND}{particle number density}
\makeacroif{FDTD}{Finite Difference Time Domain}
\makeacroif{DOIT}{Discrete Ordinate ITerative}
\makeacroif{PR}{Precipitation Radar}
%
\makeacroif{HIRS}{High-resolution Infrared Radiation Sounder}
\makeacroif{AVHRR}{Advanced Very High Resolution Radiometer}
\makeacroif{MC}{Monte Carlo}
\makeacroif{NOAA}{National Oceanic and Atmospheric Administration}
\makeacroif{ARTS}{Atmospheric Radiative Transfer Simulator}
\makepluralacroif{mpbs}{MPBS}{monochromatic pencil-beam simulation}
\makeacroif{RMSE}{root mean square error}
\makeacroif{GCM}{general circulation model}
\makeacroif{CPR}{Cloud Profiling Radar}
\makeacroif{IWP}{Ice Water Path}
\makeacroif{AMSU}{Advanced Microwave Sounding Unit}
\makeacroif{MHS}{Microwave Humidity Sounder}
\makeacroif{ANN}{Artificial Neural Network}
\makeacroif{radar}{RAdio Detection and Ranging}
\makeacroif{IR}{infrared}
\makeacroif{MW}{microwave}
\makeacroif{MODIS}{MODerate resolution Imaging Spectroradiometer}
\makeacroif{FRAC}{Full-Resolution Area Coverage}
\makeacroif{GAC}{Global Area Coverage}
\makeacroif{TIROS}{Television InfraRed Operational Sounder}
\makeacroif{SRF}{spectral response function}
\makeacroif{DOIT}{Discrete Ordinate ITerative}
\makeacroif{DOM}{Discrete Ordinate Method}
\makeacroif{ECMWF}{European Centre for Medium-range Weather Forecasting}
\makeacroif{FASCOD}{Fast Atmospheric Signature CODe}
\makeacroif{MSU}{Microwave Sounding Unit}
\makeacroif{SSM/I}{Special Sensor Microwave/Imager}
\makeacroif{TIROS}{Television Infra-Red Operational Sounder}
\makeacroif{GCM}{General Circulation Model}
\makeacroif{ERA}{\protect\gls{ECMWF} Re-Analysis}
\makeacroif{NESDIS}{National Environment Satellite, Data and Information Service}
\makeacroif{MSPPS}{Microwave Surface and Precipitation Products System}
\makeacroif{UTH}{Upper Tropospheric Humidity}
\makeacroif{PWV}{Precipitable Water Vapour}
\makeacroif{TOA}{Top Of Atmosphere}
\makeacroif{IWC}{Ice Water Content}
\makeacroif{NPP}{National Partnership Program}
\makeacroif{EOS}{Earth Observing System}
\makeacroif{ATMS}{Advanced Technology Microwave Sounder}
\makeacroif{CrIS}{Cross-track Infrared Sounder}
\makeacroif{AIRS}{Atmospheric Infrared Sounder}
\makeacroif{FOV}{Field Of View}
\makeacroif{FOR}{Field Of Regard}
\makeacroif{VIIRS}{Visible Infrared Imager Radiometer Suite}
\makeacroif{FY}{Feng-Yun}
\makeacroif{MWHS}{Micro-Wave Humidity Sounder}
\makeacroif{MWS}{Micro-Wave Sounder}
\makeacroif{SG}{Second Generation}
\makeacroif{STAR}{Center for Satellite Applications and Research}
\makeacroif{MIRS}{Microwave Integrated Retrieval System}
\makeacroif{TIROS}{Television InfraRed Operational Satellite}
\makeacroif{TOVS}{\protect\gls{TIROS} Operational Vertical Sounder}
\makeacroif{ATOVS}{Advanced \protect\gls{TIROS} Operational Vertical Sounder}
\makeacroif{AAPP}{Advanced \protect\gls{ATOVS} Processing Package}
\makeacroif{POES}{Polar Orbiting Environmental Satellites}
\makeacroif{NASA}{National Aeronautics and Space Administration}
\makeacroif{HIRLAM}{High Resolution Local Area Modelling}
\makeacroif{NWP}{Numerical Weather Prediction}
\makeacroif{HARMONIE}{\protect\gls{HIRLAM} Aladin Regional/Mesoscale Operational \protect\gls{NWP} In Europe}
\makeacroif{RTSP}{Radiative Transfer and Surface Properties}
\makeacroif{WG}{Working Group}
\makeacroif{ITWG}{International \protect\gls{TOVS} Working Group}
\makeacroif{RDR}{Raw Data Record}
\makeacroif{SDR}{Sensor Data Record}
\makeacroif{CDR}{Climate Data Record}
\makeacroif{VTPR}{Vertical Temperature Profile Radiometer}
\makeacroif{DWD}{Deutsche Wetter-Dienst}
\makeacroif{JPL}{Jet Propulsion Library}
\makeacroif{SSU}{Stratospheric Sounding Unit}
\makeacroif{C3VP}{Canadian CloudSat/CALIPSO Validation Project}
\makeacroif{SNO}{Simultaneous Nadir Observation}
\makeacroif{NRL}{Naval Research Laboratory}
\makeacroif{DARDAR}{raDAR/liDAR}
\makeacroif{CALIOP}{Cloud-Aerosol Lidar with Orthogonal Polarization}
\makeacroif{VLIDORT}{Vectorized Linearized Discrete Ordinate Radiative Transfer}
\makeacroif{DISORT}{Discrite Ordinate Radiative Transfer}
\makeacroif{LMD}{Laboratoire de M\'{e}t\'{e}orologie Dynamique}
\makeacroif{CLEX}{Cloud Layer EXperiment}
\makeacroif{CSU}{Colorado State University}
\makeacroif{IIR}{Imaging Infrared Radiometer}
\makeacroif{EUMETSAT}{European Organisation for the Exploitation of Meteorological Satellites}
\makeacroif{CM}{Climate Monitoring}
\makeacroif{SAF}{Satellite Application Facility}
\makeacroif{TWP}{Total Water Path}
\makeacroif{LERMA}{Laboratoire d'Etude du Rayonnement et de la Mati\`ere en Astrophysique}
\makeacroif{ITSC}{International \protect\gls{TOVS} Study Conference}
\makeacroif{PSD}{Particle Size Distribution}
\makeacroif{FASTEM}{Fast Emissivity (FIXME CHECK ACRONYM!)}
\makeacroif{TELSEM}{A Tool to Estimate Land Surface Emissivities at Microwave}
\makeacroif{RTTOV}{Radiative Transfer for \protect\gls{TOVS}}
\makeacroif{ASG}{Atmospheric Scene Generator}
\makeacroif{NICAM}{Nonhydrostatic ICosahedral Atmospheric Model}

\newcommand{\methodused}[1]{{\bfseries #1}}
\newcommand{\collocs}{Collocations Toolkit}

\title{\collocs\ in Atmlab (version 2-1-273)}

\date{\isodate\today}

\author{Gerrit Holl}

\setcounter{secnumdepth}{5}
\setcounter{tocdepth}{5}
\begin{document}
%\pagenumbering{Roman}
\maketitle
%\newpage
\tableofcontents
%\newpage
\listoffigures
%\newpage
\listoftables
%\newpage
\pagenumbering{arabic}

\section{Introduction}

A frequent task in analysis of geophysical data is the collocation in space
and time between data from different datasets.
For example, one might wish to collocate geolocated remote sensing measurements from
spaceborne sensors with model output, or collocate measurements from different
sensors with each other.
Usually, someone develops code in his or her favourite programming language
to tackle the collocation problem at hand.
Depending on context, the problem can vary considerably in nature and in
complexity.
Nevertheless, many sub-problems occur in a wide variety of situations.
Valuable time may be lost by people re-inventing the wheel over and over
again.

The \collocs\ is a set of classes and routines to easily
handle collocations between an arbitrary pair of (satellite) datasets.
It has been designed for user-friendly processing of long time series of data in a
wide variety of situations.

The toolkit is currently implemented in Matlab and is fully integrated with
Atmlab, a package for atmospheric data analysis.
It is available at a free license and is completely free of charge.

Practical information on how to download the code can be found at
\url{http://www.sat.ltu.se/projects/collocations/toolkit.php}.

This is a technical document, and does not contain detailed information on
algorithms used in the process of collocating.
For information on the algorithms, please refer to the descriptions in the
refereed articles \citet{holl10:_collocating_amt} and
\citet{john12:_understanding_jgr}, and in my licentiate thesis
\citep{holl11:_lic}.

This is an introductory document.
For full information on all classes and methods, please refer to the online
documentation in Matlab (best accessed with the help browser, e.g.\
type \lstinline!doc SatDataset! on the Matlab commandline).
This also contains more examples than are provided in this document.

\subsection{Overview of features}

This subsection is a work in progress.

\begin{itemize}
\item Automatically take care of duplicates between subsequent granules
through a database describing the first scanline not contained in the
previous granule.
\item Split collocations between core, fieldcopier, collapsed, where the
latter can be generated at the same time or later.
\item Cached function evaluation
\end{itemize}

\section{Getting started}

To get started with collocations:

\begin{itemize}
\item Configure your datasets (\autoref{sec:SatDataset}).
This involves:
\begin{itemize}
\item Defining where your data are stored
\item Creating a reading routine, if this does not already exist
\item Defining how the data shall be read
\end{itemize}
This is not needed if the datasets are already defined.
\item Define your CollocatedDatasets (\autoref{sec:CollocatedDataset})
\begin{itemize}
\item Define what datasets are collocated with each other
\item Define maximum distance and time to have a collocation
\item Define where the data shall be stored
\end{itemize}
This is not needed if the CollocatedDataset is already defined.
\item Define what additional data shall be processed.
\begin{itemize}
\item The easy way: via predefined-classes such as FieldCopier
and Collapser (\autoref{sec:simple}).
FieldCopiers are to copy fields from the original data to a collocated
dataset, whereas Collapsers can be used e.g.\ for averaging a small-footprint
over a large-footprint dataset, or perform another task where the data are
reduced (``collapsed'').
\item The advanced way: via user-defined subclasses of AssociatedDataset
(\autoref{sec:advanced}).
\end{itemize}
This is not needed if the FieldCopiers and Collapsers are already defined.
\item Perferm the collocations with
\lstinline|CollocatedDataset.collocate_and_store_date_range|. 
This is not needed if the collocations are already done.
\item Read the collocations with \lstinline|CollocatedDataset.read|.
\end{itemize}

\autoref{fig:cd} shows the most important classes and methods that together
form the \collocs.

\section{Defining the base datasets: SatDataset}
\label{sec:SatDataset}

Before we can start collocating, we need to be able to read data for the
datasets we want to collocate, and to systematically go through a large number of granules.
A SatDataset object contains all information needed to find granules on the
file-system given a particular date and time, to read the granule (while
removing duplicates) and to return it in a format that a CollocatedDataset object can
work with.
See the class diagram in \autoref{fig:cd} for a partial list of attributes and
methods.
For a complete list of properties, type \lstinline|properties SatDataset|
in Matlab.
For a complete list of methods, type \lstinline|methods SatDataset|.

Perhaps your dataset is already defined in Atmlab.
To check this, inspect \lstinline|datasets()|:

\begin{lstlisting}
>> D = datasets(); D.amsub

ans = 

  SatDataset handle

  Properties:
                    name: 'amsub'
                 basedir: '/storage3/data/amsu'
                  subdir: '$SAT_amsub_$YEAR4/$MONTH/$DAY'
                      re: [1x165 char]
                filename: []
                  reader: @common_read_poes_radiometer
        granule_duration: 6130
                 satname: []
                   cache: [1x1 CachedData]
                 visible: 1
                    sats: {'noaa15'  'noaa16'  'noaa17'}
                   tryre: 1
    needs_starttimesfile: 0
          starttimesfile: 'granule_start_times.mat'
              starttimes: []
                metadata: []
     collocated_datasets: [1x7 CollocatedDataset]
     starttimes_fullpath: '/storage3/data/amsu/granule_start_times.mat'

  Methods, Events, Superclasses

\end{lstlisting}

This shows the properties with their associated values.
Some properties must be defined for any dataset.
Those include in particular the name, the location where the data are
stored, and the reading routine.
If you want to collocate data, you also need to define
\lstinline|granule_duration|.
For details on what every property means, check \lstinline|help SatDataset|.
The dataset \lstinline|amsub| is already defined, as are many others.
However, you may need to define a new dataset.

\subsection{Defining a new \lstinline|SatDataset|}

If your dataset is already in Atmlab and configured for your system, you
can skip this step.
To check this, call the \lstinline|find_granules_by_date| method for a
date where you expect to have granules.
If it returns a matrix with start times, you can skip this subsection.

Defining a new \lstinline|SatDataset| is done by calling the constructor,
i.e.\ calling the class:

\begin{lstlisting}
  >> spam = SatDataset('name', 'MyDataset', 'granule_duration', '3600', ...
                       'satname', 'SpamSat', 'reader', @my_reading_routine);
\end{lstlisting}

Only the name you need to pass when you are creating the dataset.
When a SatDataset gets created, it registers itself with a global
structure.
This has two implications:
\begin{itemize}
\item Firstly, each SatDataset must have a unique name
\item Secondly, a SatDataset is persistent.
Whether you bind it to a name directly or not at all, exactly one
SatDataset gets created
\end{itemize}
Also note that SatDataset descends from \lstinline|handle|.
Therefore, it does not pass by value, but passes by reference.

For all other properties, you can define them in two ways:

\begin{itemize}
\item Immediately upon creation.
This is done by passing property names and values in pairs to
\lstinline|SatDataset|, exactly as when you would create a structure.
\item After creation.
For this, the syntax is again like for structures:
\lstinline|spam.basedir = '/path/to/file'|.
\end{itemize}

Datasets that are built-in in Atmlab define dataset properties in two
steps.
In \lstinline|define_datasets|, the universal properties are defined;
those that are the same no matter what machine one runs on.
This includes the name, the reading routine, and some other properties.
In \lstinline|datasets_init|, the site-specific properties are defined;
those that differ based on the site, such as the path to the data.
You may wish to follow a similar pattern.

Now suppose we have a new dataset called \lstinline|ssmt2|\footnote{In
reality, this too comes with Atmlab; but let's pretend it doesn't}.
This we can do in the following steps:

\begin{enumerate}
\item Define the dataset:

\begin{lstlisting}
SatDataset('name', 'ssmt2', ...
    'needs_starttimesfile', false, ...
    'reader', @common_read_ssmt2, ...
    'granule_duration', 7000);
\end{lstlisting}

For the meaning of each of the properties, please refer to the in-Matlab
documentation, i.e.\ \lstinline|help SatDataset/granule_duration|.
For examples, you can look at the source code for
\lstinline|define_datasets.m|.

\item
One property however, is of particular importance, and that is the
\lstinline|reader|.
This points to a reading routine with a prescribed interface: input and
output arguments \emph{must} match a particular pattern that is documented
by \lstinline|help SatDataset/reader|.

If there is no appropriate reading routine, create it.
Use \lstinline|help SatDataset/reader| and existing
\lstinline|common_read_*| routines as an example.
\item
Define where the data are:

\begin{lstlisting}
D.ssmt2.basedir = '/storage3/data/ssmt_ngdc';
D.ssmt2.subdir = '$SAT/$YEAR4/$MONTH/$DAY';
D.ssmt2.filename = '$SAT$YEAR4$MONTH$DAY$HOUR$MINUTE.T2.gz';
D.ssmt2.re = '(?<satname>F[0-9]{2})(?<year>\d{4})(?<month>\d{2})(?<day>\d{2})(?<hour>\d{2})(?<minute>\d{2})\.T2\.gz';
\end{lstlisting}
 
It is essential that you define at least \lstinline|basedir|,
\lstinline|subdir|, and at least one of \lstinline|filename| and
\lstinline|re|.
Those properties are used to locate granules in a universal manner.
Check the respective property documentation for details.
\end{enumerate}

Now you are reading to use the \lstinline|SatDataset| as documented in
\autoref{sec:capa_sat}.

\subsection{Capabilities of \lstinline|SatDataset|}
\label{sec:capa_sat}

\begin{itemize}
\item Locate granules. The methods \lstinline!find_datadir_by_date!,
\lstinline|find_granules_by_date|, \lstinline|find_granules_for_period|,
and \lstinline|find_granule_by_datetime|
locate granules on the file system for a particular date and time.
This is done using the attributes \lstinline|basedir|, \lstinline|subdir|,
\lstinline|re|, and \lstinline|filename|.
This functionality is essential for finding collocations, but also useful in
its own right.
For example, \lstinline|find_granules_for_period| is useful for time series
analysis.
An example of the capabilities of \lstinline|find_granules_by_date|:
\begin{lstlisting}
>> D.ssmt2.find_granules_by_date([2001, 1, 10], 'F12')

ans =

        2001           1           9          23          19          -1
        2001           1          10           1           1          -1
        2001           1          10           2          43          -1
        2001           1          10           4          25          -1
        2001           1          10           6           7          -1
        2001           1          10           7          49          -1
        2001           1          10           9          31          -1
        2001           1          10          11          12          -1
        2001           1          10          12          54          -1
        2001           1          10          14          36          -1
        2001           1          10          16          18          -1
        2001           1          10          18           0          -1
        2001           1          10          19          42          -1
        2001           1          10          21          24          -1
        2001           1          10          23           6          -1


paths = 

  Columns 1 through 12

    [1x61 char]    [1x61 char]    [1x61 char]    [1x61 char]    [1x61 char]    [1x61 char]    [1x61 char]    [1x61 char]    [1x61 char]    [1x61 char]    [1x61 char]    [1x61 char]

  Columns 13 through 15

    [1x61 char]    [1x61 char]    [1x61 char]

\end{lstlisting}
\item Collect and find information on duplicates.
Before you can actually read any data, a database needs to be created that
documents the number of lines in a particular granule that are repetitions
of the granule before.
For many datasets, subsequent granules contain duplicates: a number of
scanlines are repeated.
The method \lstinline|find_granule_first_line| creates a hashtable describing
for each granule what the first scanline is that does not occur in the
previous scanline, and stores this to a file.
%Then, \lstinline|granule_first_line| looks this up from the hashtable.
Reading routines in the \collocs\ use this information by default to guarantee
no duplicates occur in final products.
\item Actually read the data.
This is done with the method \lstinline|read_granule|.
In order to make this work, a tailor-made reading routine needs to be provided
via the property \lstinline|reader|, passed as a function handle.
This routine provides the interface between the data as stored on disc (which
can be in an arbitrary format) and the data as represented inside the
\collocs.
Detailed documentation can be found with \lstinline|help SatDataset/reader|.
An example:
\begin{lstlisting}
>> data = D.ssmt2.read_granule(grans(1, :), 'F12')
19-Feb-2013 11:57:24.809:SatDataset.read_granule:818:Reading /storage3/data/ssmt_ngdc/F12/2001/01/09/F12200101092319.T2.gz
19-Feb-2013 11:57:24.864:exec_system_cmd:64:/local/gerrit/python2.7/bin/python2.7 /storage4/home/gerrit/checkouts/atmlab/sensors/ssmt2_reader_netcdf.py /local/ramdisk/gerrit/F12200101092319.T2 /local/ramdisk/gerrit/tp4d2a7ee8_74a4_4070_91e3_9673628956d8.nc
19-Feb-2013 11:57:25.608:exec_system_cmd:69:Warning: didn't find number of data records directly.
Using number of records - number of header records (764 - 1) instead
19-Feb-2013 11:57:25.640:SatDataset.granule_first_line:940:Reading /storage3/data/granules_firstline/firstline_F12_ssmt2.mat. (If this fails, run self.find_granule_first_line (SatDataset.find_granule_first_line)).
19-Feb-2013 11:57:25.813:CachedData.set_entry:78:Setting cache entry: ssmt2F120109_TueJan231810 (new size: 4 entries, 4.913 MiB)

data = 

                  lat: [763x28 single]
                  lon: [763x28 single]
           ancil_data: [763x10 double]
    global_attributes: [1x1 struct]
                 time: [763x1 double]
                epoch: 978998400
              version: 'v2.1b'
      scanline_number: [763x28 double]
    scanline_position: [763x28 double]
\end{lstlisting}
\end{itemize}

% Unless you want to re-implement the collocation algorithm, there is no need
% to subclass SatDataset; all the functionality you will need is contained in
% the class.

\section{Defining the collocated datasets: CollocatedDataset}
\label{sec:CollocatedDataset}

When the SatDatasets are defined, you can create the CollocatedDataset.
As a subclass of SatDataset, a CollocatedDataset contains all
functionality of the SatDataset, plus more.
The core properties uniquely essential to CollocatedDataset are:

\begin{description}
\item[primary] The primary SatDataset to be collocated
\item[secondary] The SatDataset that the primary is collocated with
\item[distance] The maximum distance (in km) to be allowed
\item[interval] The maximum interval (in s) to be allowed
\end{description}

Again, details in the documentation for the respective property
(\lstinline|help CollocatedDataset/distance| etc.).

\subsection{Creating a CollocatedDataset}

To create a CollocatedDataset, call the constructor as below.
The first two arguments refer to the primary and the secondary datasets to
be collocated, respectively.
The remaining arguments are as for SatDataset, but you must set the
\lstinline|distance| and \lstinline|interval| properties.
Note that there is a decent default value for \lstinline|reader|, so there
is no need to define this for a CollocatedDataset (but it will anyway only
be used if you use collocate a CollocatedDataset with another SatDataset).

\begin{lstlisting}
>> CollocatedDataset(D.ssmt2, D.ssmt2, 'name', 'cd_ssmt2', 'distance', 30, 'interval', 900)

ans = 

  CollocatedDataset handle

  Properties:
                 primary: [1x1 SatDataset]
               secondary: [1x1 SatDataset]
                distance: 30
                interval: 900
                gridsize: 2
                 logfile: 'collocations_log'
                  marker: '------------------------------------------------------------------------\n'
              associated: {}
           members_const: [1x1 struct]
                    cols: [1x1 struct]
                 members: [1x1 struct]
               overwrite: 0
                     pcd: []
                 mattype: 'double'
                    name: 'cd_ssmt2'
                 basedir: []
                  subdir: []
                      re: []
                filename: []
                  reader: @(varargin)self.read_homemade_granule(varargin{:})
        granule_duration: 86400
                 satname: []
                   cache: [1x1 CachedData]
                 visible: 1
                    sats: []
                   tryre: 1
    needs_starttimesfile: 0
          starttimesfile: 'granule_start_times.mat'
              starttimes: []
                metadata: []
     collocated_datasets: []
     starttimes_fullpath: 'granule_start_times.mat'
\end{lstlisting}

Remember that CollocatedDataset is a subclass of SatDataset, so this too
gets created and permanently registered just by calling the constructor.

A CollocatedDataset consists of two SatDatasets.
The CollocatedDataset derives from SatDataset.
That has some implications:
\begin{itemize}
\item Anything that works for SatDataset, also works for CollocatedDataset:
list granules, find duplicates (normally none), actually read the data, etc.
\item Since a CollocatedDataset consists of two SatDatasets, one or two of
those may actually be CollocatedDatasets themselves.
Therefore, you can arbitrarily deeply collocate CollocatedDatasets with
other CollocatedDatasets or SatDatasets.
\end{itemize}

\subsection{Using a CollocatedDataset}

Beyond the abilities of a SatDataset, a CollocatedDataset performs the
following tasks:
\begin{itemize}
\item Collocations between granules, as described by
\citet{holl10:_collocating_amt} and subsequent papers.
To collocate single granules or days without storing the result, you can
use \lstinline|collocate_granule| and \lstinline|collocate_date|.
You will not need the low-level method \lstinline|collocate|, but you
might want to tweak the value of the property \lstinline|gridsize|; see
\lstinline|help CollocatedDataset/gridsize| depending on the expected
number of collocations. As a rule of thumb: if you expect more collocations,
choose a small gridsize; if you expect few, choose a large one.
Note that this only affects the speed and not the results as such.
\item Storing those collocations in an appropiate format\footnote{This is
actually implemented in the parent-class HomemadeDataset, but as an end-user
you don't need to worry about this}.
To collocate a single day, use \lstinline|collocate_and_store_date|.
To collocate a long range of dates, use
\lstinline|collocate_and_store_date_range|.
If you have any AssociatedDatasets (see below), you can pass them along
either at the same time or at a later date.
In the latter case, the toolkit will realise the core collocations already
exist.
\item Reading those collocations at a later date, possibly along with
appropiate AssociatedDatasets (FieldCopiers, Collapsers, etc.).
This is done with the \lstinline|read| method. 
For example:
\begin{lstlisting}
  [M, c] = ...
        col.read([2007 1 1],[2007 1 10], 'n18', ...
                 {'LAT1', 'LON1', 'LAT2', 'LON2', 'RO_ice_water_path', 'cld_reff_vis','cld_opd_vis'}, ...
                  struct('LAT1', [-30 30]), ...
                  {{@(x, y) x>y, {'LAT1', 'LON1'}}});
\end{lstlisting}
For detailed help, please consult the in-Matlab documentation for
\lstinline|CollocatedDataset/read|.
\end{itemize}

For full information on how to create a CollocatedDataset object, the
properties and the methods, use \lstinline|doc CollocatedDataset| on the
Matlab command-line.

\subsection{Defining additional data}

Core collocations contain only the bare information, that applies always:
scanline numbers, scanline indices, latitude, longitude, times for both
instruments, distance, time interval, and a running number.
For most applications, this is not sufficient: one also wants a subset of the
data from the original datasets, or even geophysical data calculated from this
original data.
To this end, collocation methods (like
\lstinline|collocate_and_store_date_range|) and reading methods (such as
\lstinline|read|) take as as argument a collection of additional objects known
as AssociatedDatasets.
The AssociatedDataset is an abstract class.
This means that it defines for a number of methods that the inputs and output
are, but it does not actually implement the methods.
It also described what properties the class has.
Two implementations are provided (see \autoref{sec:simple}).
Those contain at least all the methods and properties defined in AssociatedDataset,
and it is in the class AssociatedDataset that those are documented.

Implementations of AssociatedDataset have their objects created in different ways, but used
similarly.
When instantiated, the objects are registered with the CollocatedDataset to which
they belong.
Then, when e.g.\ \lstinline|collocate_and_store_date_range| is called for this
dataset, the user can specify what AssociatedDatasets should be generated along
with it.
When reading collocations, there is no need to specify the
AssociatedDatasets, as long as all the fieldnames are unique.
The reading routine figures out automatically from what dataset (the core, or one
of the AssociatedDatasets) a particular field originates.

\subsubsection{The simple way: FieldCopier and Collapser}
\label{sec:simple}

FieldCopier and Collapser are both implementations of AssociatedDataset.
The usage of most (but not all) methods is identical.
Therefore, most of the documentation is in AssociatedDataset, and not in
FieldCopier or Collapser.

\paragraph{Fieldcopier}

With a FieldCopier, one can do exactly what the name says: copy fields exactly
from the original data to the collocated data.
Like the CollocatedDataset, the FieldCopier derives from SatDataset%
\footnote{In an indirect way: as illustrated by \autoref{fig:cd}, FieldCopier
implements AssociatedDataset, AssociatedDataset derives from HomemadeDataset,
and HomemadeDataset derives from SatDataset.
One could say that SatDataset is the great-grandparent of FieldCopier}, so it
too can list granules, it contains the same properties to define where the data
will be stored, etc.
With CollocatedDataset it shares the capability that it can store data (via
the intermediate class HomemadeDataset).

\subparagraph{FieldCopier features}

The FieldCopier can:

\begin{itemize}
\item Copy fields from the datasets that were originally collocated.
For example, one collocates MHS with CPR\_2B\_CWC\_RO, then obtains the
field ROIWP from the secondary.
\item Copy fields from \emph{sibling datasets}.
A sibling dataset is a dataset for the same instrument but in different
granules.
For example, the different CPR products all have the same granules, with
the same number of measurements in each granule.
To get a field from a different dataset, set the structure member.
\lstinline|dataset|; see \lstinline|FieldCopier.fieldstruct_primary|.
\item Copy fields from a different instrument on the same dataset.
For example, for the aforementioned collocations, one might want to obtain
AMSU-A data.
The granules have the same starting times, but the number of measurements
is the same.
This is a bit more involved, because the user has to define how to make
the translation, for example, from MHS coordinates (scanline number,
scanline position) to AMSU-A coordinates.
This is implemented in the abstract class
\lstinline|FieldMultiInstrumentCopier|.
For example, see its implementation \lstinline|AssociatedPOESPlusCPR|.
\end{itemize}

For detailed information, consult the online documentation
(\lstinline|doc FieldCopier| on the Matlab command line).

\subparagraph{Creating a FieldCopier --- defining the members structure}

The actual creation of a FieldCopier is analogous to the creation of
CollocatedDataset, SatDataset, etc.; see the documentation for
\lstinline|FieldCopier/FieldCopier| for details.

The most involved part of creating a FieldCopier is to define the
\lstinline|members| structure.
See the documentation for the FieldCopier class and examples in
\lstinline|define_datasets| for details.

\subparagraph{Using a FieldCopier}

Pass it on wherever you can pass an AssociatedDataset

\paragraph{Collapser}

A Collapser can be used for collocations between datasets with different
footprint sizes.
For example, it may be used to calculate the mean, the standard deviation, and
other statistics, for CloudSat \gls{CPR} footprints over larger \gls{MHS}
footprints.
For detailed information, see \lstinline|doc Collapser|.

\subparagraph{Creating a Collapser}

When creating a Collapser, you can define processors and limitators.
See the documentation for the Collapser class and examples in
\lstinline|define_datasets| for details.

\subparagraph{Using a Collapser}

\paragraph{Others}

Another, specialised implementation is the class
\lstinline|AssociatedPOESPlusCPR|.

\subsubsection{The advanced way: rolling your own class}
\label{sec:advanced}

The tasks performed by FieldCopier and Collapser cover many situations, but
far from all.
For the greatest flexibility, you can create your own implementation of
AssociatedDataset.
This requires to create a subclass from AssociatedDataset, similar to
FieldCopier and Collapser, and implement at least all attributes and methods
that are abstract in AssociatedDataset.
Then, an object of this class can be passed on in the same place as
FieldCopier, Collapser, or any other implementations.

The best documentation to date is the full class documentation and the source
code for AssociatedDataset, FieldCopier, Collapser, and other implementations
for AssociatedDataset.

\section{Class diagram}

\begin{figure}
\begin{tikzpicture}[every node/.style={font=\scriptsize}]
 \begin{class}{SatDataset}{0, 0}
  \attribute{+ name}
  \attribute{+ basedir}
  \attribute{+ subdir}
  \attribute{+ re}
  \attribute{+ filename}
  \attribute{+ reader : function\_handle}
  \attribute{+ granule\_duration : double}
  \attribute{+ satname}
  \attribute{+ dataset}
  %\attribute{+ firstline\_filename}
  \attribute{\# collocated\_dataset}
  \operation{+ find\_datadir\_by\_date}
  \operation{+ find\_info\_from\_granule}
  \operation{+ \methodused{find\_granules\_by\_date}}
  \operation{+ find\_granule\_by\_datetime}
  \operation{+ \methodused{find\_granules\_for\_period}}
  \operation{+ read\_granule}
  \operation{+ granule\_first\_line}
  \operation{+ find\_granule\_first\_line}
 \end{class}

 \begin{class}{HomemadeDataset}{1, -7}
  \inherit{SatDataset}
  \attribute{+ cols}
  \operation{+ store}
  \operation{\# read\_single\_day}
 \end{class}

 \begin{class}{CollocatedDataset}{-5, -10}
 \inherit{HomemadeDataset}
 \attribute{+ primary}
 \attribute{+ secondary}
 \attribute{+ distance}
 \attribute{+ interval}
 \attribute{\# associated}
 \attribute{\# members}
 \operation{+ overlap\_granule}
 \operation{+ collocate}
 \operation{+ collocate\_granule}
 \operation{+ \methodused{collocate\_date}}
 \operation{+ collocate\_and\_store\_date}
 \operation{+ \methodused{collocate\_and\_store\_date\_range}}
 \operation{+ process}
 \operation{+ read}
 \end{class}

 \begin{abstractclass}{AssociatedDataset}{3, -10}
 \inherit{HomemadeDataset}
 \attribute{+ dependencies}
 \operation{+ members2cols}
 \operation{+ process\_delayed}
 \operation{+ merge}
 \operation{+ get\_mergefields}
 \operation{+ concatenate}
 \operation[0]{+ primary\_arguments}
 \operation[0]{+ secondary\_arguments}
 \operation[0]{+ needs\_primary\_data}
 \operation[0]{+ needs\_secondary\_data}
 \operation[0]{+ process\_granule}
 \end{abstractclass}

 \begin{class}{FieldCopier}{-3, -15}
 \implement{AssociatedDataset}
 \end{class}

 \begin{class}{Collapser}{4, -15}
 \implement{AssociatedDataset}
 \end{class}

 \unidirectionalAssociation{Collapser}{collects}{}{FieldCopier}
 \association{CollocatedDataset}{consists of two}{}{SatDataset}{can have many}{}


\end{tikzpicture}
\caption{Diagram of the most important classes and methods in the toolkit.
Methods that an end-user will commonly call are shown \methodused{like this}.
This is a class diagram \citep{wiki:cd}.
\label{fig:cd}%
}
\end{figure}

\clearpage
\section{Other stuff}

\printglossaries

\bibliographystyle{copernicus}
\bibliography{bib_all,extra}

\end{document}
