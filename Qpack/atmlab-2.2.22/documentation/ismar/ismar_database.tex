%% LyX 2.1.2 created this file.  For more info, see http://www.lyx.org/.
%% Do not edit unless you really know what you are doing.
\documentclass[english]{scrreprt}
\usepackage[T1]{fontenc}
\usepackage[latin9]{inputenc}
\setcounter{secnumdepth}{-2}
\setcounter{tocdepth}{3}
\usepackage{textcomp}
\usepackage{url}
\usepackage{amsmath}

\makeatletter

%%%%%%%%%%%%%%%%%%%%%%%%%%%%%% LyX specific LaTeX commands.
%% Because html converters don't know tabularnewline
\providecommand{\tabularnewline}{\\}

%%%%%%%%%%%%%%%%%%%%%%%%%%%%%% User specified LaTeX commands.
\usepackage{chngcntr}
\counterwithout{figure}{chapter}
\counterwithout{table}{chapter} 

\makeatother

\usepackage{babel}
\begin{document}

\title{ISMAR Retrieval Database}


\author{Manfred Brath}

\maketitle

\section{Database description}

The database consists of radiative transfer simulations of $1\text{D}$-atmospheric
profiles made with ARTS 2.2%
\footnote{ARTS 2.3 is still under strong development with a lot of changes,therefore
the database will be defined according to ARTS 2.2%
}. As the database should be the foundation of a retrieval, the database
has to be consistent and self-standing, meaning each state of the
database should be comparable with each other. So, following assumptions
are made:
\begin{itemize}
\item All the different states of the database should be on common grids
of pressure coordinates and looking direction.
\item All states should have the same types of hydrometeors, hydrometeor
sizes and shapes.
\item All states should have the same spectral composition.
\end{itemize}
Therefore, the database has a static part with the grids, instrument
and scatterer definitions and a variable part with the state variables
changing for each case. So, the database consists of $6$ grids, $2$
instrument description, $5$ descriptions for the scatterers and $13$
state variables, which all will be described in the folowing parts.


\subsection{Static part}


\subsubsection*{Grids }
\begin{itemize}
\item $case\_index$, an index that names a specific case with a number.
\item $pressure$, the vertical grid. It is as first try defined from $1000\,\textrm{hPa}$
to $30\,\textrm{hPa}$ with $75$ logarithmic distributed steps. 
\item $channel\_no$, the index of the channel, see table \ref{tab:Channel-definition}
for the description.
\item $looking\_direction$, the angular grid. It is as first try defined
from $0\text{\textdegree}$ to $180\text{\textdegree}$ in steps of
$5\text{\textdegree}$ ($37$ steps).
\item $scattering\_element\_index$, an index that names a specific scattering\_element
with a number.
\item $polarization$, a four element grid stating the polarization state.
The first component is the $I$-component, the second is the $Q$-component,
the second is the $U$-component and the second is the $V$-component
of the Stokes vector. 
\end{itemize}

\subsubsection*{Instrument description}
\begin{itemize}
\item $channel\_def\left(channel\_no\right)$, the channel description,
it consists of the center frequencies, the offsets of the sidebands
and the bandwidths, see table \ref{tab:Channel-definition}.
\begin{table}
\begin{centering}
\begin{tabular}{c|r@{\extracolsep{0pt}.}l|r@{\extracolsep{0pt}.}l|r@{\extracolsep{0pt}.}l|c}
 & \multicolumn{2}{c|}{Center fre- } & \multicolumn{2}{c|}{Side} & \multicolumn{2}{c|}{Band-} & \tabularnewline
$channel\_no$ & \multicolumn{2}{c|}{quency {[}$\text{GHz}${]}} & \multicolumn{2}{c|}{bands {[}$\text{GHz}${]}} & \multicolumn{2}{c|}{widths {[}$\text{GHz}${]}} & Instrument\tabularnewline
\hline 
1 & 23&8 & $\ \pm$0&07 & $\negmedspace$0&127 & Deimos\tabularnewline
\hline 
2 & 50&1 & $\pm$0&08 & 0&082 & Deimos\tabularnewline
\hline 
3 & 89&0 & $\pm$1&1 & 0&65 & MARSS\tabularnewline
\hline 
4 & 118&75 & $\pm$1&1 & 0&4 & ISMAR\tabularnewline
\hline 
5 & $\ $118&75 & $\pm$1&5 & 0&4 & ISMAR\tabularnewline
\hline 
6 & 118&75 & $\pm$2&1 & 0&8 & ISMAR\tabularnewline
\hline 
7 & 118&75 & $\pm$3&0 & 1&0 & ISMAR\tabularnewline
\hline 
8 & 118&75 & $\pm$5&0 & 2&0 & ISMAR\tabularnewline
\hline 
9 & 157&05 & $\pm$2&6 & 2&6 & MARSS\tabularnewline
\hline 
10 & 183&31 & $\pm$1&0 & 0&45 & MARSS\tabularnewline
\hline 
11 & 183&31 & $\pm$3&0 & 1&0 & MARSS\tabularnewline
\hline 
12 & 183&31 & $\pm$7&0 & 2&0 & MARSS\tabularnewline
\hline 
13 & 243&20 & $\pm$2&5 & 3&0 & ISMAR\tabularnewline
\hline 
14 & 325&15 & $\pm$1&5 & 1&6 & ISMAR\tabularnewline
\hline 
15 & 325&15 & $\pm$3&5 & 2&4 & ISMAR\tabularnewline
\hline 
16 & 325&15 & $\pm$9&5 & 3&0 & ISMAR\tabularnewline
\hline 
17 & 424&70 & $\pm$1&0 & 0&4 & ISMAR\tabularnewline
\hline 
18 & 424&70 & $\pm$1&5 & 0&6 & ISMAR\tabularnewline
\hline 
19 & 424&70 & $\pm$4&0 & 1&0 & ISMAR\tabularnewline
\hline 
20 & 448&0 & $\pm$1&4 & 1&2 & ISMAR\tabularnewline
\hline 
21 & 448&0 & $\pm$3&0 & 2&0 & ISMAR\tabularnewline
\hline 
22 & 448&0 & $\pm$7&2 & 3&0 & ISMAR\tabularnewline
\hline 
23 & 664&0 & $\pm$4&2 & 3&0 & ISMAR\tabularnewline
\hline 
24 & 874&4 & $\pm$6&0 & 3&0 & ISMAR\tabularnewline
\hline 
\end{tabular}
\par\end{centering}

\protect\caption{\label{tab:Channel-definition}Channel description, taken from: \protect\url{http://www.sat.ltu.se/workshops/ismar/material/20140611_11_rule_ismar_overview.pdf}}
\end{table}

\item $calibration$, the used calibration, it is a string with either ``PlanckBT''
or ``RayleighJeansBT'' indicating if the brightness temperature
is calibrated using the Planck equation or using the Rayleigh-Jeans
approximation. 
\end{itemize}

\subsubsection*{Scatterer descriptions}
\begin{itemize}
\item the properties of the scattering elements: 

\begin{itemize}
\item mass $\left[\text{kg}\right]$: $mass(scattering\_element\_index)$ 
\item maximum dimension $\left[\text{m}\right]$:$dmax(scattering\_element\_index)$
\item volume $\left[\text{m}^{3}\right]$:$volume(scattering\_element\_index)$ 
\item type of scattering elemenent\\
$scattering\_element\_type(scattering\_element\_index)$, an cell
array with a tag for each $scattering\_element\_index$ according
to the arts 2.2 method \textquotedbl{}ParticleSpeciesSet\textquotedbl{}
\item shape of the scattering element\\
$scattering\_element\_shape(scattering\_element\_index)$, an cell
array with a stringe for each $scattering\_element\_index$ describing
the shape.
\end{itemize}
\end{itemize}

\subsection*{Variable part}


\subsubsection*{State variables}
\begin{itemize}
\item the atmospheric data:

\begin{itemize}
\item temperature$\left[\text{K}\right]$: $temperature(case\_index,\, pressure)$ 
\item altitude $\left[\text{m}\right]$:$atltitude(case\_index,\, pressure)$ 
\item particle number density $\left[\text{m}^{-3}\right]$:\\
 $pnd\_field(case\_index,\, pressure,\, scattering\_element\_index)$ 
\item water vapor as volume mixing ratio $x\left(case\_index,\, pressure\right)$
\item brightness temperature $\left[\text{K}\right]$:\\
 $t\_b(case\_index,\, pressure,\, channel\_no,\, looking\_direction,\, polarization)$.
It is calculated by using the Planck equation or using the Rayleigh-Jeans
approximation, which is shown by the variable $calibration$. If the
Rayleigh-Jeans approximation is used, the conversion from specific
intensity to brightness temperature will be done seperately for each
component and for each simulated frequency before the mapping to the
channels will be done. If planck is used only the $I$ component will
be converted to brightness temperature. The other components will
be set to zero, because they are not defined for Planck brightness
temperature. The mapping is done by taking the average brightness
temperature of the frequencies belonging to the specific channel.
\end{itemize}
\item the surface properties 

\begin{itemize}
\item surface altitude~$\left[\text{m}\right]$: $surface\_altitude(case\_index)$
\item surface temperature $\left[\text{K}\right]$: $surface\_temperature(case\_index)$ 
\item surface type $surface\_type(case\_index)$, a cell array with a string
defining the surface type of each $case\_index$. 
\item surface reflectivity
\[
\begin{gathered}surface\_reflectivities(case\_index,\, channel\_no,\\
\, looking\_direction,\, polarization,\, polarization)
\end{gathered}
\]
the two last dimensions are the $4\times4$ reflection matrices for
a four dimensional stokes vector with the above mentioned components
\end{itemize}
\item the simulated position in space and time 

\begin{itemize}
\item latittude $\left[\text{\textdegree}\right]$: $lat(case\_index)$ 
\item longitude $\left[\text{\textdegree}\right]$: $lon(case\_index)$
\item time $time(case\_index)$\end{itemize}
\end{itemize}

\end{document}
