%% 
%% The Qpack2 manual
%%

\documentclass[a4paper,11pt]{article}


\usepackage{verbatim}
\usepackage{url}

%- Page layout
%
\pagestyle{empty}
\usepackage[top=30mm,left=30mm,right=30mm,bottom=30mm,nohead]{geometry}


\title{Qpack2 -- a Matlab tool for performing\\atmospheric 
       retrievals of OEM type}
\author{Patrick Eriksson\\(\texttt{patrick.eriksson@chalmers.se})}
\date{\today}


\newcommand{\newfield}{\vspace{1ex}\noindent}


%------------------------------------------------------------------------------
\begin{document}


\maketitle

\section{Scope}
\label{sec:scope}
%---
Qpack2 is a retrieval system, implemented in Matlab, for performing inversions
of atmospheric observations inside the framework of ``optimal estimation''
(OEM). With good knowledge of the background systems a wide range of retrievals
can be handled, but the primary aim is to provide relatively
straightforward retrievals of vertical profiles of atmospheric quantities. That
is, the atmosphere is treated to be of ``1D type'' during the retrieval.
Accordingly, a main application areas of Qpack2 is ground-based observations.

The package should be general and flexible for covered measurements. For
example, there is basically no limitations regarding observation geometry. Each
measurement can consist of several spectra, obtained by some scanning procedure
or different instruments. See Sec.~\ref{sec:scanning} for a discussion of
possibilities and limitations when it comes to handle more complex
measurements. Further, Qpack2 has been developed to be suitable for operational
inversions, as long as not most extreme calculation speed is required. Some
important aspects here are that batch calculations are allowed and atmospheric
a priori profiles can automatically be extracted from climatology data.



\section{Background and introduction}
\label{sec:background}
%---
Qpack2 is part of the Atmlab package of Matlab functions. In fact, it is
largely a merge of some of the systems in Atmlab. The computational engine,
i.e.\ the forward model, of Qpack2 is ARTS-2 ({\it Eriksson et al.}, 2011).
ARTS is a C++ program, downloaded separately (i.e.\ not part of Atmlab). The
communication with ARTS-2 is made through a systen denoted as Qarts. OEM
inversions are performed by the function \verb|oem.m|. Climatology data are
stored and interpolated through a data format called ``atmdata''.

This can be seen as a direct successor of Qpack ({\it Eriksson et al.}, 2005),
despite all code is written from scratch. For both Qpack versions the
computations are controlled by a set of settings fields packed into a structure
denoted as Q (though Qpack2 has also a structure O). Qpack was built around
ARTS-1, this new Qpack version is mainly a consequence of that ARTS-2 is the
maintained version of ARTS. (ARTS-2 is below just denoted as ARTS.) This gave
also an opportunity to make a more stringent implementation and to add some
features. This should hopefully make it easier to maintain and extend Qpack2.
Data formats are now more clearly defined and the ambition level around
documentation is higher (but still modest). A main improvement compared to
Qpack is that extraction of a priori data from climatology data is now an
integrated part. On the other hand, the feature of classifying errors into
different categories (0-3 in Qpack) and the associated plotting features were
removed as these features are dificult to implement in a generic way.

Quality checks are not handled by Qpack2. For example, there is no default
check that the frequencies of the measurements are as expected. This for
efficiency reasons and the fact that such checks are hard to make in a
totally general manner. Checks are left for accompanying dedicated
functions, such as \verb|qp2_check_f|.


\section{Software and installation}
\label{sec:download}
%---
The needed software packages are ARTS and Atmlab. The simplest is to obtain
and update these packages through svn. Download instructions for ARTS are
found at \url{www.sat.ltu.se/arts/getarts}. The installation details are here
not repeated. A plain installation of ARTS suffices.

Download options for Atmlab are found at \url{www.sat.ltu.se/arts/tools}. The
file CONFIGURE (found in the top folder) gives instructions for how to get
started with Atmlab. Qarts2 demands that several ``atmlab settings'' are
activated. This is handled by the function \verb|atmlab|. A description of all
atmlab settings is obtained by:
\begin{verbatim}
>> help atmlab
\end{verbatim}
At least the following atmlab settings are used: \verb|ARTS_PATH|,
\verb|ARTS_INCLUDES|, \verb|VERBOSITY|, \verb|FMODEL_VERBOSITY|,
\verb|STRICT_ASSERT| and \verb|WORK_AREA|. Some example settings:
\begin{verbatim}
atmlab( 'ARTS_PATH',          fullfile(homedir,'ARTS/arts/src/arts') );
atmlab( 'ARTS_INCLUDES',      fullfile(homedir,'ARTS/arts/includes') );
atmlab( 'FMODEL_VERBOSITY',   0                                      ); 
atmlab( 'VERBOSITY',          1                                      );
atmlab( 'STRICT_ASSERT',      true                                   );
atmlab( 'WORK_AREA',          '/tmp'                                 );
\end{verbatim}
The standard choice is to place these calls of \verb|atmlab| in
\verb|atmlab_conf.m|, as described in CONFIGURE.


\section{Documentation}
\label{sec:doc}
%---
The overall documentation is found in this document. The detailed information
is mainly stored in the implementation files. Information on individual
functions is obtained by the standard Matlab help command. For example:
\begin{verbatim}
>> help oem
\end{verbatim}
Some of the used data structures are documented matching the requirements of
\verb|qinfo.m|. For example, to list all fields and the documentation text of
Qarts' \verb|Q|:
\begin{verbatim}
>> qinfo(@qarts)
\end{verbatim}
The documentation for a specific field is obtained as
\begin{verbatim}
>> qinfo(@qarts,'F_BACKEND')
\end{verbatim}
Wildcards are allowed when defining fields:
\begin{verbatim}
>> qinfo(@qarts,'ABS_*')
\end{verbatim}
Details around development and bug fixes are described in the ChangeLog file of
Atmlab (\verb|atmlab/ChangeLog|).



\section{Overview}
\label{sec:overview}
%---

\subsection{Practicalities}
%
A retrieval by Qpack2 has three main steps:
\begin{verbatim}
% 1: Define forward model and retrieval settings
[Q,O] = my_q_fun;

% 2: Import the measurement data to be inverted
Y = my_y_fun;

% 3: Perform the inversion
L2 = qpack2( Q, O, Y );
\end{verbatim}
The variables \verb|Q|, \verb|O|, \verb|Y| and \verb|L2| are all structures,
and the fields of these structures are described in Section \ref{sec:structs}.

The main part of the settings defined in step 1 are found in \verb|Q|, that
match directly the \verb|Q| of the Qarts interface to ARTS. All forward model
variables are part of \verb|Q|, and retrieval variables such as grids and a
priori data are also defined in \verb|Q|. The exceptions are settings directly
associated with OEM, that are stored in \verb|O|. This division reflects the
fact that the OEM operations are made by a stand-alone function. This function,
\verb|oem|, is implemented in a general manner, to be applicable for any
retrieval case as long as an interface to a ``forward model'' is provided.

The measurement data structure \verb|Y| can be an array. That is, a series of
observations can be inverted in a single call of \verb|qpack2|, on the
condition that a \verb|Q| and \verb|O| are valid for all observations. This
implies e.g.\ that basic sensor characteristics such as backend frequencies
must be common for all the data cases in \verb|Y|. On the other hand, both
observation position and direction can differ between the observations, and the
data from e.g.\ an aircraft flight could be inverted in a single run.

The usage of the functions \verb|my_q_fun| and \verb|my_y_fun| in steps
1 and 2 is just a suggestion for how to organise the definition phase of the
run, but it is probably a good idea to separate the inversion settings and the
task of importing measurement data.

When \verb|Q|, \verb|O| and \verb|Y| are created, it is just to call
\verb|qpack2|, and everything should work automatically. The input of
\verb|qpack2| is fixed (throughout \verb|Q|, \verb|O| and \verb|Y|). The output
is normally the result of the retrieval, packed as a ``level 2'' data
structure (\verb|L2|). The function can also provide simulated measurements.
This option is triggered by setting the spectrum field to be empty (if \verb|Y|
is an array, this must be done for all elements):
\begin{verbatim}
...
Ye.Y = []; 
Ysim = qpack2( Q, O, Ye );
\end{verbatim}
The output is a copy of \verb|Ye|, with the field \verb|Ysim.Y| set to the
result of the forward model run. The simulated data match the a priori state of
the retrieval set-up.

Several full examples are found in Atmlab's folder \verb|demos|, such as
\verb|qpack2_demo.m|. The examples work with simulated data, created by
\verb|qpack2| itself. The \verb|qpack2_demo.m| example file is also appended to
this document as Appendix \ref{app:qfun}. A related example is
\verb|arts_oem_demo|. It does not use Qpack2, but shows how an OEM inversion is
performed with ARTS as the forward model, and the calculation steps in this
example gives a good view of the interior of Qpack2.


\subsection{Units}
\label{sec:units}
%
In most cases SI units are used. For example, frequency off-sets are retrieved
in Hz. For retrieval of species profiles several ``units'' can be used. First
of all, both volume mixing ratio and number density retrievals can be made.
These options are selected by setting the species unit to \verb|vmr| and
\verb|nd|, respectively.

Further, the species unit can be also be set to \verb|rel|. In this case, the
retrieved profile is expressed in fractions of the a priori profile. For
example, the a priori state corresponds to 1, and $x=2$ signifies that the
species amount is double as high as the a priori. The remaining species unit is
\verb|logrel|, that is defined as the natural logarithm of the \verb|rel| unit.
The main reason for using this unit is to introduce a constrain of positive
species values. For example, if z is a retrieved \verb|logrel| value and $v_a$
is the a priori VMR, the retrieved volume mixing ratio is $v_a\exp(z)$,
which always is greater than 0.

The prescribed or selected unit is used for all related variables. For example,
a priori covariance matrices must be specified correspondingly. For example, if
the selected species unit is \verb|rel|, the 1 standard deviation a priori
uncertainty should be in the order of 0.5 (that corresponds to a 50\%
uncertainty). The same applies to output data. No automatic unit conversions
are performed (in contrast to the earlier Qpack version), e.g. the returned a
priori profile for \verb|logrel| retrievals is a vector of zeros. The basic
idea is to return data that are fully consistent with the assumptions of the
retrieval.

Data retrieved using the \verb|rel| and \verb|logrel| units can be converted to
volume mixing ratio data by the function \verb|qp2_rel2vmr|. However, this
function handles only data directly associated with the species, it does not
cover, for example, the Jacobian matrix. See further the help text of the
function.
 


\section{Main data structures}
\label{sec:structs}

\subsection{Forward model and retrieval settings}
\label{sec:Q}
%---
\subsubsection{General features}
Forward model parameters and most retrieval variables are joined into a single
structure, denoted as \verb|Q|. This structure has a fixed set of fields. That
is, the structure must always contain exactly this set of fields. However, some
fields are used only in particularly circumstances and not all fields must be
set. A field of \verb|Q| is flagged as undefined by setting it to \verb|{}|,
and e.g.\ \verb|[]| is taken as an active selection. This applies also to
structures \verb|O| and \verb|Y|.

The structure \verb|Q| of Qpack2 is identical to the \verb|Q| of Qarts. Qpack2
makes only slight modifications of \verb|Q| before it is given to Qarts. The
Qarts system is primarily an interface to the ARTS forward model, on the same
time as retrieval variables can be defined in a relatively general manner.
However, Qarts is not a complete retrieval system, it just allows definition of
retrieval variables in parallel with the forward model data. The ambition is
just to provide the basis for retrieval systems, such as Qpack2 or streamlined
software dedicated to a single sensor.

Qarts is documented through the qinfo feature:
\begin{verbatim}
>> qinfo(@qarts)
\end{verbatim}
The call of \verb|qinfo| above provides a description of all defined fields
(Sec.\ \ref{sec:download}) and this information is also found here as
Appendix~\ref{app:q}. It is recommended to initialise \verb|Q| as
\begin{verbatim}
Q = qarts;
\end{verbatim}
This ensures that Q contains all defined fields, also ones introduced more
lately. 

All features of ARTS can be accessed through Qarts. Calculations of standard
type can largely be handled by defining the fields of Qarts that have a direct
matching ARTS workspace variables (WSV). That is, the name is the same in Qarts
and ARTS, beside that Qarts use uppercase for its fields and ARTS lowercase for
its variables (e.g.\ \verb|Z_SURFACE| and \verb|z_surface|, respectively).
Qarts has also several fields that allow inclusion of workspace method (WSM)
calls. Some of these fields match ARTS agendas and have then a dedicated
purpose (e.g.\ \verb|SURFACE_RTPROP_AGENDA|). There are also fields where you are
free to include any set of WSM calls (e.g.\ \verb|WSMS_BEFORE_RTE|). You can
also use ARTS's include feature through the field \verb|INCLUDES|.

Accordingly, Qarts is a relatively clean and open interface to ARTS, and to
set-up a calculation demands primarily an understanding of ARTS. Built-in
documentation of ARTS WSMs and WSVs is obtained in a terminal as
\begin{verbatim}
$ arts -d z_surface
\end{verbatim}
or from within Matlab as
\begin{verbatim}
>> arts('-d z_surface');
\end{verbatim}
See the documentation of ARTS for more in-depth documentation. 

The best introduction to the practical usage of ARTS is obtained through the
control file examples found in ARTS's folder \verb|controlfiles|. For an
introduction of the usage of ARTS through Qarts, there are several example
scripts in Atmlab's folder \verb|demos|, such as \verb|qarts_demo|.


\subsubsection{Qpack2 specifics}
%
Most fields of \verb|Q| can be unspecified (\verb|{}|), but Qpack2 handles some
fields in special way:
\begin{itemize}
\item \verb|Q.RAW_ATMOSPHERE| must be unset (\verb|{}|).
\item All existing ``atmdata''-fields are considered and are given highest
  priority. That is, if such a field is set, any setting of the corresponding
  field will be overwritten. For example, if \verb|Q.T.ATMDATA| is set, any
  existing values in \verb|T_FIELD| will have no effect.
\item The fields \verb|T_FIELD|, \verb|VMR_FIELD| and \verb|Z_FIELD| must
  be defined directly in \verb|Q|, possibly through the corresponding atmdata
  fields (see point above). That is, these workspace variables can not
  be set by an include file. 
\item In the case of \verb|Z_FIELD| there is another option, this field can be
  set by applying hydrostatic equilibrium, then activated by setting
  \verb|Q.HSE.ON| to true. The reference point for hydrostatic equilibrium is
  specified by \verb|Y.HSE_P| and \verb|Y.HSE_Z| (Sec.~\ref{sec:Y}). 
\item \verb|Q.HSE_P| is set to \verb|Y.HSE_P|. Other fields of \verb|Q.HSE|
  must be set by the user, thus including an active choice for \verb|Q.HSE.ON|.
\item \verb|Q.ABSORPTION| must be set. The option 'CalcTable' is not allowed
  (as it is difficult to ensure generation of good absorption tables in a
  general manner). You must either use 'OnTheFly' or create an absorption table
  as a pre-calculation (the 'LoadTable' option).
\item \verb|Q.TNOISE_C| must be set.
\item The a priori knowledge covariance matrix (Sx) is created by
  \verb|arts_sx| and the \verb|SX| sub-fields of the retrieval quantities must
  be set.
\end{itemize}



\subsection{OEM settings}
\label{sec:O}
%---
Settings directly associated with OEM are put into the structure \verb|O|. Also
this setting structure is documented through \verb|qinfo| (see also Appendix
\ref {app:o}):
\begin{verbatim}
>> qinfo(@oem)
\end{verbatim}
Many of the fields of \verb|O| controls the output of the \verb|oem| function.
For Qpack2 you do not need to set these fields manually. The output required to
provide the specified L2 data (Section \ref{sec:l2}) is ensured by initialising
\verb|O| as
\begin{verbatim}
O = qp2_l2( Q );
\end{verbatim}
Some fields of O are also set by Qpack2, and user settings are overwritten.
These fields are:
\begin{itemize}
\item \verb|O.msg|
\item \verb|O.sxnorm| 
\end{itemize}
For linear inversion, a single active setting is needed:
\begin{verbatim}
O.linear = true;
\end{verbatim}



\subsection{Measurement data}
\label{sec:Y}
%---
The series of measurements to be treated is packed into a
data structure denoted as \verb|Y|. The data fields of this structure are
described in \verb|qp2_y|, following the \verb|qinfo| format:
\begin{verbatim}
>> qinfo(@qp2_y)
\end{verbatim}
The same information is appended to this document in
Appendix~\ref{app:y}. An example:
\begin{verbatim}
>> Y(3)

ans = 

           DAY: 3
             F: [800x1 double]
          HOUR: 11
         HSE_P: 10000
         HSE_Z: 1.5677e+04
      LATITUDE: 57.4000
     LONGITUDE: 11.9300
        MINUTE: 57
         MONTH: 3
        SECOND: 58.0000
        TNOISE: 0.0243
             Y: [800x1 double]
          YEAR: 2002
            ZA: 65
    Z_PLATFORM: 5
\end{verbatim}
The simplest way to initialise \verb|Y| is to use \verb|qp2_y| as
\begin{verbatim}
for i = 1 : n
  Y(i) = qp2_y;
  load( file{i} )
  Y(i).Y = ...
end
\end{verbatim}


\subsection{Retrieved data}
\label{sec:l2}
%---
Retrieved data are returned as a structure array \verb|L2|. The fields of
\verb|L2| are not fixed, the set of returned fields depend on settings in
\verb|Q|. This makes it impossible to use qinfo here (as for \verb|Q| and
\verb|Y|), and the fields are instead described below in this document.

The only mandatory field of \verb|L2| is \verb|converged|. The remaining
content of L2 is controlled by the Qarts field \verb|L2_EXTRA| and the
\verb|L2| sub-field for the retrieval quantities (e.g.\ \verb|Q.POLYFIT.L2|).
As an example on a complete \verb|L2|, the first element of the output of
\verb|qpack2_demo| is (the polyfit part is truncated):
\begin{verbatim}
>> L2(1)

ans = 

             year: 2008
            month: 2
              day: 25
             hour: {}
           minute: {}
           second: {}
        converged: 1
               dx: 4.4374e-06
             cost: 1.0748
           cost_x: 5.8390e-04
           cost_y: 1.0742
                f: [1279x1 double]
                y: [1279x1 double]
               yf: [1279x1 double]
               bl: [1279x1 double]
           p_grid: [45x1 double]
          t_field: [45x1 double]
          z_field: [45x1 double]
    species1_name: 'O3'
       species1_p: [45x1 double]
      species1_xa: [45x1 double]
       species1_x: [45x1 double]
       species1_e: [45x1 double]
      species1_eo: [45x1 double]
      species1_es: [45x1 double]
      species1_mr: [45x1 double]
          ffit_xa: 0
           ffit_x: -3.8104e+03
           ffit_e: 8.7798e+03
          ffit_eo: 8.6418e+03
          ffit_es: 1.5504e+03
          ffit_mr: 0.9692
      polyfit0_xa: 0
       polyfit0_x: 0.0516
       polyfit0_e: 0.4748
      polyfit0_eo: 0.0967
      polyfit0_es: 0.4649
      polyfit0_mr: 0.7745
      polyfit1_xa: 0
       polyfit1_x: -0.0022
       polyfit1_e: 0.0097
      polyfit1_eo: 0.0094
      polyfit1_es: 0.0023
      polyfit1_mr: 0.9996
      ...
\end{verbatim}
A description of the options for \verb|Q.L2_EXTRA| is obtained by:
\begin{verbatim}
>> help qp2_l2
\end{verbatim}
Further comments and description of fields directly associated with specific
retrieval quantities are found below (no ambition of completeness here).


\subsubsection{Various general fields}
%
\newfield 
\verb|cost|, \verb|cost_x| and \verb|cost_y|:\\
Cost values of retrieved state and fit to measurement. See further \verb|oem.m|.
All obtained by including the string \rq cost\rq\ in \verb|L2_EXTRA|.

\newfield 
\verb|P.year|, \verb|P.month|, \verb|P.day|, \verb|P.hour|, \verb|P.minute|,
and \verb|P.second|:\\
The time of the measurement. 
All obtained by including the string \rq date\rq\ in \verb|L2_EXTRA|.

\newfield 
\verb|tnoise|:\\
Mean of assumed thermal noise. This is simple average, no weighting
with channel widths is applied. This value has no direct physical
meaning, but can be used as a mean to e.g. remove the most noisy
measurements (without having \verb|Y| at hand). 
Obtained by including the  string \rq tnoise\rq\ in \verb|L2_EXTRA|.

\newfield 
\verb|f|:\\
Frequency vector of measurement. Equals \verb|Y.F| if set. Otherwise
set \verb|Q.SENSOR_RESPONSE_F| or \verb|Q.F_GRID|, depending on if a sensor is
applied or not. Only included together with \verb|y|.

\newfield 
\verb|y|:\\
Measured spectrum. 
Obtained by including the string \rq y\rq\ in \verb|L2_EXTRA|.

\newfield 
\verb|yf|:\\
Fitted spectrum. That is, the spectrum matching retrieved state vector.
Obtained by including the string \rq yf\rq\ in \verb|L2_EXTRA|.

\newfield
\verb|bl|:\\
Retrieved ``baseline''. That is, retrieved spectrum distortion through
e.g.\ polynomial fitting. A scalar 0 if no baseline retrieval is performed.
Obtained by including the string \rq bl\rq\ in \verb|L2_EXTRA|.

\newfield 
\verb|converged|:\\
Convergence flag. See further \verb|oem.m|. Always included.

\newfield 
\verb|dx|:\\
Change in the state vector x between each iteration. See further \verb|oem.m|.
Obtained by including the string \rq dx\rq\ in \verb|L2_EXTRA|.

\newfield
\verb|J|, \verb|G| and \verb|A|:\\
The (full) Jacobian, gain and averaging kernel matrix, respectively.
Obtained by including the string \rq J\rq, \rq G\rq\ and \rq Afull\rq,
respectively, in \verb|L2_EXTRA|. Note that the string \rq\ A\rq\ only triggers
inclusion of partial averaging kernels (see below).

\newfield
\verb|S|, \verb|So| and \verb|Ss|:\\
The covariance matrix of total, observation and smoothing error, respectively.
Obtained by including the string \rq S\rq, \rq So\rq\ and \rq Ss\rq,
respectively, in \verb|L2_EXTRA|. 

\newfield
\verb|jq| and \verb|ji|:\\
Information describing the different retrieval quantities. The structure
\verb|jq| is basically matching the user settings. One example:
\begin{verbatim}
>> L2(1).jq{1}

ans = 

         maintag: 'Absorption species'
          subtag: 'O3'
            mode: 'vmr'
      analytical: 1
    perturbation: 0
           grids: {[48x1 double]}
\end{verbatim}
The second structure, \verb|ji|, gives the start and end index in the state
vector for each retrieval quantity. The data are stored in a manner
matching the data type used in ARTS for this information, an ArrayOfIndex. 
For example, if retrieval quantity two holds position 49 in 96 in the state
vector, we have that
\begin{verbatim}
>> L2(1).ji{2}

ans = 

    [49]    [96]
\end{verbatim}
These two structures are included automatically when \verb|A|, \verb|S| or any
data matching the complete state vector is included in the output, and, hence,
the structures are needed match vector and matrix elements to the different
retrieval quantities.



\subsubsection{Background atmospheric state}
%
\newfield \verb|p_grid|, (\verb|lat_grid|, \verb|lon_grid|), \verb|t_field| and
\verb|z_field|:\\
As the ARTS workspace variables with the same name. The data are identical to
the ones that are provided to the forward model. (If temperatures are retrieved,
this information is used. Otherwise returned data match climatology
temperatures.)
Obtained by including the string \rq ptz\rq\ in \verb|L2_EXTRA|.


\subsubsection{Retrieved species}
%
The retrieved species are indexed after the order they are specified in
\verb|Q| and named as \verb|species1|, \verb|species2| \dots A number of fields
is returned for each species. The field names below are valid for species 1. If
two species are retrieved there is also a field named as \verb|species2_name|
etc. The unit of output follows \verb|Q.SPECIES(i).UNIT|, see further
Section~\ref{sec:units}. These set of fields are available:

\newfield 
\verb|species1_name|:\\
Tag name of species.

\newfield 
\verb|species1_p|:\\
Vertical/pressure retrieval grid for the species.

\newfield 
\verb|species1_x|:\\
Retrieved profile. Unit depends on \verb|Q.SPECIES(i).UNIT|.

\newfield 
\verb|species1_xa|:\\
A priori state for the species. Unit depends on \verb|Q.SPECIES(i).UNIT|.
Obtained by including the string \rq xa\rq\ in \verb|L2_EXTRA|.

\newfield 
\verb|species1_e|:\\
Total retrieval (observation + smoothing) error for the species. 
Unit depends on \verb|Q.SPECIES(i).UNIT|.
Obtained by including the string \rq e\rq\ in \verb|L2_EXTRA|.

\newfield 
\verb|species1_eo|:\\
Observation error for the species. Unit depends on \verb|Q.SPECIES(i).UNIT|.
Obtained by including the string \rq eo\rq\ in \verb|L2_EXTRA|.

\newfield 
\verb|species1_es|:\\
Smoothing error for the species. Unit depends on \verb|Q.SPECIES(i).UNIT|.
Obtained by including the string \rq es\rq\ in \verb|L2_EXTRA|.

\newfield
\verb|species1_mr|:\\
Measurement response for the species. This is the sum of the rows of
\verb|species_A| (see below).
Obtained by including the string \rq mresp\rq\ in \verb|L2_EXTRA|.

\newfield
\verb|species1_A|:\\
The species specific averaging kernels. That is, this is the diagonal
centred sub-matrix covering only the resolution between changes of the
species at different altitudes. The resolution between variables of
different retrieval quantities is thus not covered.
Obtained by including the string \rq A\rq\ in \verb|L2_EXTRA|.

\newfield
\verb|species1_vmr0|:\\
Added automatically if the retrieval unit is \verb|rel| or \verb|logrel|, to be
used to convert data to VMR. For example, the retrieved profile in VMR for the
\verb|rel| case is \verb|x.*vmr0|.



\subsubsection{Baseline fit variables}
%
There are two approaches for performing baseline fits, denoted as
\verb|POLYFIT| and \verb|SINEFIT|. The two approaches can be used 
seperately, or be combined.

A set of fields is added for each polynomial coefficient, and
for the first polynomial coefficient they are

\newfield \verb|P.polyfit0_x|, \verb|P.polyfit0_xa|,
\verb|P.polyfit0_e|,
\verb|P.polyfit0_eo|, \verb|P.polyfit0_es| and\\
\verb|P.polyfit0_mr|:\\
As corresponding fields above for species (\verb|species1_x| etc.).\\

\noindent
Fields are added in a similar manner for \verb|SINEFIT|. For the first 
selected ripple period the fields are

\newfield \verb|P.sinefit1_x|, \verb|P.sinefit1_xa|,
\verb|P.sinefit1_e|,
\verb|P.sinefit1_eo|, \verb|P.sinefit1_es| and\\
\verb|P.sinefit1_mr|:\\
These fields have length 2, where the two values correspond to the cosine and
sine terms, respectively. 


\subsubsection{Frequency fit variables}
%
Both a shift and a stretch factor can be retrieved:

\newfield \verb|P.fshift_x|, \verb|P.fshift_xa|,
\verb|P.fshift_e|,
\verb|P.fshift_eo|, \verb|P.fshift_es| and\\\verb|P.fshift_mr|:\\
As corresponding fields above for species (\verb|species1_x| etc.).

\newfield \verb|P.fstretch_x|, \verb|P.fstretch_xa|,
\verb|P.fstretch_e|,
\verb|P.fstretch_eo|, \verb|P.fstretch_es| and\\\verb|P.fstretch_mr|:\\
As corresponding fields above for species (\verb|species1_x| etc.).



\section{Data types and file formats}
\label{sec:formats}

\subsection{Qpack2 variables}
%
The data types in Qpack2 (and in Qarts) follow largely ARTS. Regarding basic
data types, there are two main things to consider. Firstly, a distinction is
made between the logical 0 and 1, and the integers 0 and 1. Accordingly,
booleans, sometime also denoted as flags, must be set to \verb|true| or
\verb|false|. Secondly, vectors are demanded to be columns.

A notable feature of Qpack2 and Qarts is that data fields, booleans and scalars
excluded, can either be set directly or be set to the name of a file (including
data of the expected type). Both these settings are possible
\begin{verbatim}
Q.F_GRID = [ 501.18e9 : 1e6 : 501.58e9 ]';
Q.F_GRID = 'f_odinsmr_ac2a.xml;
\end{verbatim}
Note the transpose in the first case (to create a column vector).

The fields \verb|Q.ABS_SPECIES.ATMDATA|, \verb|Q.T.ATMDATA| and
\verb|Q.Z_ATMDATA| are intended for data of climatology character. The data
format used here is called atmdata. The format is similar to the GriddedField
format in ARTS, but has also fields for e.g.\ units. The atmdata format is
based on the general gformat:
\begin{verbatim}
>> help gformat
\end{verbatim}
For a description of the specialities of atmdata:
\begin{verbatim}
>> help isatmdata
\end{verbatim}
For example, the MSIS temperature climatology found in arts-xml-data is stored
as GriddedField4 but imported as
\begin{verbatim}
Q.T_ATMDATA = gf_artsxml( fullfile( arts_xmldata_path, 'climatology', ...
                        'msis90', 'msis90.t.xml' ), 'Temperature', 't_field' );
\end{verbatim}
This produces data of atmdata type:
\begin{verbatim}
>> Q.T_ATMDATA

ans = 

          TYPE: 'atmdata'
          NAME: 'Temperature'
        SOURCE: [1x65 char]
           DIM: 4
          DATA: [4-D double]
     DATA_NAME: 'Temperature'
     DATA_UNIT: 'K'
         GRID1: [141x1 double]
    GRID1_NAME: 'Pressure'
    GRID1_UNIT: 'Pa'
         GRID2: [19x1 double]
    GRID2_NAME: 'Latitude'
    GRID2_UNIT: 'deg'
         GRID3: 0
    GRID3_NAME: 'Longitude'
    GRID3_UNIT: 'deg'
         GRID4: [13x1 double]
    GRID4_NAME: 'doy'
    GRID4_UNIT: ''
\end{verbatim}



\subsection{File formats}
%
The standard format for data input files is ``ARTS xml''. These files are
hardly created by hand, but stored from Matlab (through \verb|xmlStore|) or
from ARTS (through \verb|WriteXML|). If ARTS is compiled with support for
NetCDF, that format can also be used (\verb|arts_nc_write_|{\it datatype} and
\verb|WriteNetCDF|, respectively).



\section{Measurements consisting of multiple spectra}
\label{sec:scanning}
%
As long as each measurement consists of a single spectrum, Qpack2 should be
quite flexible in treating sensor aspects. For some conditions, it also
possible to handle measurements consisting of multiple spectra. 

\subsection{Common sensor characteristics}
%
If the sensor characteristics are common for all spectra involved (not
including thermal noise) it is relatively straightforward to apply Qpack. For
example, the same sensor is moved vertically, or rotated to measure at
different zenith angles, can be treated. Requirements and considerations to
handle such multiple-spectra measurements include:
\begin{itemize}
\item The fields \verb|Y.Z_PLATFORM| and \verb|Y.ZA| shall give the altitude
  and zenith angle of each spectrum, and are accordingly (column) vectors
  instead of scalar values for single-spectra cases. 
\item \verb|Y.LATITUDE| and \verb|Y.LONGITUDE| are kept as scalar values, the
  same geographical position is assumed for all spectra.
\item \verb|Y.T_NOISE| must be extended where each column describes the noise
  of the corresponding spectrum. Hence, the noise level can vary between the
  spectra.
\item If a baseline fit is performed, the corresponding \verb|SX|-fields must
  be extended. These fields are scalar values for single-spectrum
  observations. With multiple spectra, the covariance matrices can include an
  expected correlation between the baseline coefficient for different spectra.
  For example, if the zero-order \verb|POLYFIT| coefficient is expected to have
  a standard deviation of 2\,K and has a correlation of 0.9 between two
  spectra, these assumptions are included as
\begin{verbatim}
Q.POLYFIT.SX0 = 2.0^2 * [ 1 0.9; 0.9 1 ];
\end{verbatim}
\end{itemize}
It is also possible to describe the scanning in a 3D atmosphere, that should be
of interest if the azimuth angle matters. This option should be of interest for
cases including a Zeeman influence or wind retrievals. Additional
considerations are then:
\begin{itemize}
\item \verb|Q.ATMOSPHERE_DIM| shall be set to 3.
\item \verb|Q.LAT_GRID| and \verb|Q.LON_GRID| must be specified.
\item Atmospheric fields, such as \verb|Q.T_FIELD|, must be 3D. The standard
  choice should be to set the fields to be constant in latitude and longitude
  directions. Using the \verb|ATMDATA| feature with an input 1D atmosphere
  handles this automatically.
\item The field \verb|Y.AA| shall be set, and match \verb|Y.ZA| in length.
\item Latitude and longitude retrieval grids, such in
  \verb|Q.ABS_SPECIES.GRIDS|, must be specified, but must have length one. Any
  value should work. One example:
\begin{verbatim}
Q.ABS_SPECIES(1).GRIDS = { Q.P_GRID, mean(Q.LAT_GRID), mean(Q.LON_GRID)};
\end{verbatim}
Retrieved atmospheric data will effectively have no latitude and longitude
dimension and be of 1D type.
\end{itemize}
The file \verb|qpack2_wind3d_demo| exemplifies retrievals of this type.


\subsection{More complex set-ups}
%
It can be considered to append spectra from different instruments to perform a
joint retrieval. In a few cases this can be handled as in the sub-section
above, but would normally require a more complex set-up. Another more complex
situation is when the sensor characteristics change during e.g.\ a limb
scanning sequence. Both these examples can be handled by ARTS by making use of
the \verb|yCalcAppend| workspace method.

Using Qpack together with \verb|yCalcAppend| is not impossible but requires
detailed knowledge of how both Qpack and ARTS work and far from all possible
measurement scenarios can be treated. Hence, it is best to contact Patrick
for advice before start trying to implement a set-up of this type.




\section*{References}
Eriksson, P., C. Jim\'enez, and S. A. Buehler, Qpack, a general tool for
instrument simulation and retrieval work, {\it J. Quant. Spectrosc. Radiat.
  Transfer}, 91, 47-64, 2005. \vspace{1ex}

\noindent
Eriksson, P., S. A. Buehler and C. P. Davis and C. Emde and O. Lemke, 
ARTS, the atmospheric radiative transfer simulator, Version 2, 
{\it J. Quant. Spectrosc. Radiat. Transfer}, 112, 1551-1558, 2011.





\appendix

\newpage\noindent{\huge \bf Appendices}

\newpage
\section{An example script}
\label{app:qfun}
%
\verbatiminput{../../demos/qpack2_demo.m}


\newpage
\section{The fields of Q}
\label{app:q}
%
The information below is obtained in Atmlab by typing:
\begin{verbatim}
>> qinfo( @qarts )
\end{verbatim}
\vspace{2ex}
\verbatiminput{qarts.txt}


\newpage
\section{The fields of O}
\label{app:o}
%
The information below is obtained in Atmlab by typing:
\begin{verbatim}
>> qinfo( @oem )
\end{verbatim}
\vspace{2ex}
\verbatiminput{oem.txt}


\newpage
\section{The fields of Y}
\label{app:y}
%
The information below is obtained in Atmlab by typing:
\begin{verbatim}
>> qinfo( @qp2_y )
\end{verbatim}
\vspace{2ex}
\verbatiminput{qp2_y.txt}





\end{document}

