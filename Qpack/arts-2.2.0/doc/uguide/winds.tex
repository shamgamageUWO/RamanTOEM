\chapter{Doppler effects and winds}
 \label{sec:winds}


 \starthistory
 130223 & Restructured to include effects beside winds (PE).\\
 121218 & First version by Patrick Eriksson.\\
 \stophistory

%\graphicspath{{Figs/rte/}}

 The default assumption in ARTS can be expressed as the atmosphere is assumed
 to be static and the observation platform is static during each measurement,
 while any relative movement between the atmosphere and the sensor will cause
 Doppler effects. ARTS handles three sources to Doppler
 shifts: winds, rotation of the planet and sensor velocity.


\section{Winds}
%==============================================================================
\label{sec:winds:defs}

Atmospheric transport is not considered by ARTS, but winds can still be of
importance due to the \textindex{Doppler effect} they can cause. This effect is
most significant at high altitudes where the line shape is narrow, and a
frequency shift of absorption and emission is most easily discerned.

The workspace variables to specify winds are \wsvindex{wind\_u\_field},
\wsvindex{wind\_v\_field} and \wsvindex{wind\_w\_field}, below denoted as
\WindWE, \WindSN\ and \WindVe, respectively. The user need to set all these
three variables. It is allowed, for all three wind components, to set the
variable to be empty, which is shorthand for saying that the wind is zero
throughout the atmosphere (Sec.~\ref{sec:atm:vecfields}). Otherwise, the size
of the variable is required to match the atmospheric grids.

No further input is required, a Doppler shift is added as soon as any of the
winds is non-zero (exceptions discussed in
Sec.~\ref{sec:winds:limitations}). For clarity, even though a setting of
\builtindoc{wind\_u\_field} always is demanded, this wind component has no
effect on Doppler shifts for 1D and 2D calculations, as the wind moves at an
angle of 90$^\circ$ from the observation plane (Sec.~\ref{sec:atm:vecfields}).


\section{Planet rotation}
%==============================================================================

ARTS applies an Earth-Centred, Earth-Fixed, (ECEF) coordinate system. This
implies that a ground-based receiver follows the planet's rotation. On the
other hand, if e.g.\ the sensor is placed on another planet, or is in a transit
orbit, the rotation of the observed planet will cause Doppler effects. This is
treated in ARTS by a short-cut, the rotational movement can be translated to an
imaginary wind by the method \wsmindex{wind\_u\_fieldIncludePlanetRotation}.


\section{Sensor velocity}
%==============================================================================

The feature is not yet fully implemented, but a rudimentary inclusion of the
sensor's velocity can be made by the \wsvindex{rte\_alonglos\_v} workspace
variable. The variable shall be set to the sensor velocity component along the
viewing direction. For the moment, this velocity is assumed to be the same for
all pencil beam calculations.


\section{Limitations}
\label{sec:winds:limitations}
%
For efficiency reasons, when extracting particle extinction (absorption and
scattering) the mean Doppler shift along the propagation path is applied (but
the shift varies with frequency). This allows to at least include larger shifts
caused by e.g. satellite velocity. The smaller shifts (due to winds) are here
of less importance as the particle extinction is quite smooth with frequency.

The above is valid for transmission type calculations. Doppler shifts are so
far totally neglected inside the DOIT and MC scattering modules.



\section{Equations}
%==============================================================================
\label{sec:winds:eqs}

The main equations for deriving the Doppler shift from the winds are given in
this section. The total wind, \Wind, is
\begin{equation}
  \Wind = \sqrt{\WindWE^2+\WindSN^2+\WindVe^2}.
  \label{eq:winds:total}
\end{equation}
The zenith angle of the wind direction is 
\begin{equation}
  \aZntAng{\Wind} = \arccos(\WindVe/\Wind),
  \label{eq:winds:za}
\end{equation}
and the azimuth angle is 
\begin{equation}
  \aAzmAng{\Wind} = \arctan(\WindWE/\WindSN). \qquad 
                           \mathrm{(implemented\ by\ the\ atan2\ function)}
  \label{eq:winds:aa}
\end{equation}
The cosine of the angle between the wind vector and the line-of-sight is
\begin{equation}
  \cos\gamma = \cos\aZntAng{\Wind}\cos\aZntAng{l} + 
               \sin\aZntAng{\Wind}\sin\aZntAng{l}
               \cos(\aAzmAng{\Wind}-\aAzmAng{\l}),
  \label{eq:winds:dang}
\end{equation}
where \aZntAng{l}\ and \aAzmAng{l}\ are the angles of the line-of-sight. 

Finally, as the winds do not reach relativistic values, the Doppler shift can
be calculated as
\begin{equation}
  \Delta\Frq = \frac{-\Wind\cos\gamma\aFrq{0}}{\speedoflight},
\end{equation}
where \aFrq{0}\ is the rest frequency and \speedoflight\ is the speed of light.
The core part of these calculations is implemented in the genral internal
function \funcindex{dotprod\_with\_los}. The negative sign 
is caused by the fact the ``line-of-sight'' is the observation direction, not
the direction of the EM waves.



