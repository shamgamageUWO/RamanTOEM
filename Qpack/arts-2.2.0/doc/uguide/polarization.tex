\chapter{Polarisation and Stokes parameters}
 \label{sec:polarization}


\starthistory
  040524 & Section on scattering matrices added by Patrick Eriksson. \\
  040426 & Created and written by Christian Melsheimer.\\
\stophistory


\graphicspath{{Figs/polarization/}}



%=====================================================================
% Definition of new commands:
% ====================================================================
% a 3-elem. column vector:
\newcommand{\ColVct}[3]{\left( \begin{array}{c}
                                   #1 \\ #2 \\ #3
                               \end{array} \right) }
% partial derivative of #1 with respect to #2, written as a fraction:
\newcommand{\PrtDrv}[2]{\frac{\partial #1}{\partial #2}}
%\newcommand{\half} {\ensuremath{\textstyle\frac{1}{2}}}
%Inverse Wave Impendance
%\newcommand{\InvImp} %
%      {\ensuremath{\sqrt{\textstyle{\frac{\epsilon}{\mu}}}}}
% various polarization basis vectors:
\newcommand{\eVrt} {\ensuremath{\VctStl{e}_v}}
\newcommand{\eHor} {\ensuremath{\VctStl{e}_h}}
\newcommand{\eLh} {\ensuremath{\VctStl{e}_{LH}}}
\newcommand{\eRh} {\ensuremath{\VctStl{e}_{RH}}}
\newcommand{\ePls} {\ensuremath{\VctStl{e}_{+45\degree}}}
\newcommand{\eMin} {\ensuremath{\VctStl{e}_{-45\degree}}}
\newcommand{\mi}  {\ensuremath{\mathrm{i}}}


The present version of ARTS implements the radiative transfer equation
in tensor form, i.e., for the 4 components of the Stokes vector, not
just for its first component, the intensity or radiance.
This means that the model can include polarisation dependence in
absorption or scattering processes.
It is therefore necessary to give some details on the polarisation of
radiation, the definition of
the Stokes parameters, and the definition of antenna polarisation.


\section{Polarisation directions}
%=====================================================================
\label{sec:polarization:directions}
Electromagnetic waves in homogeneous, isotropic media are transverse
waves, i.e., their oscillating electric and magnetic fields are in a
plane perpendicular to the propagation direction. The choice of two
basis vectors -- we shall call them polarisation directions here --
that span that transverse plane is arbitrary; often they are called
``horizontal'' and ``vertical'' and correspond to some horizontal and
vertical direction of the particular setting. Nevertheless, what is
meant by horizontal/vertical, or parallel/perpendicular, is purely a
matter of definition.

Here, we stick to the system called laboratory frame or fixed frame, used by
\citet{Mishchenko:02}: We use a coordinate system where the $z$-axis points
toward local zenith. We denote the propagation direction of radiation by a unit
vector $\VctStl{n} = \VctStl{k}/k$, where $k$ is the wave number. $\VctStl{n}$
is given by two angles, the zenith angle $\theta$ , i.e., the angle between
$\VctStl{n}$ and the $z$-axis, and the azimuth angle $\phi$, i.e., the angle
between the projection of $\VctStl{n}$ into the $xy$-plane and the $x$-axis:
\begin{equation}
  \label{eq:polarization:propdir}
   \VctStl{n} = \ColVct{\cos\phi \sin\theta}%
                       {\sin\phi \sin\theta}%
                       { \cos\theta}                    
\end{equation}
Then we define the polarisation directions by the partial derivatives
of  $\VctStl{n}$ with respect to $\theta$ and $\phi$. We shall call
them $\theta$-direction (also: vertical) and $\phi$-direction (also:
horizontal), respectively, see
Figure \ref{fig:polarization:directions}. 
\begin{figure}
 \begin{center}
  \begin{minipage}[c]{0.9\textwidth}
   \begin{center}
    \includegraphics*[width=0.9\hsize]{pol_directions}
   \end{center}
  \end{minipage}
  \begin{minipage}[c]{0.9\textwidth}
   \caption{The definition of the polarisation directions, adapted
     from \citet{Mishchenko:02}}
   \label{fig:polarization:directions}
  \end{minipage}
 \end{center}
\end{figure}   
%
Their unit basis vectors are
\begin{equation}
  \label{eq:polarization:etheta}
   \VctStl{e}_\theta = \eVrt =
    \PrtDrv{\VctStl{n}}{\theta} \left/ 
     \left\| \PrtDrv{\VctStl{n}}{\theta}\right\| \right.  
    =
    \ColVct{\cos\phi \cos\theta}%
           {\sin\phi \cos\theta}%
           { -\sin\theta}
\end{equation}
%
\begin{equation}
  \label{eq:polarization:ephi}
   \VctStl{e}_\phi = \eHor = 
    \PrtDrv{\VctStl{n}}{\phi} \left/ 
          \left\|\PrtDrv{\VctStl{n}}{\phi}\right\| \right.  
    =
    \ColVct{-\sin\phi}%
           {\cos\phi}%
           { 0}
\end{equation}
The vectors \VctStl{n}, $\VctStl{e}_\theta$ (=\eVrt),
$\VctStl{e}_\phi$ (=\eHor) are
mutually orthogonal and define (in the mentioned order) a right-handed
system, i.e., 
$\left(\VctStl{n}\times \VctStl{e}_\theta\right) \cdot
\VctStl{e}_\phi = 1$ and the same for all cyclic permutations.


\section{Plane monochromatic waves}
%=====================================================================
\label{sec:polarization:monochrom}

Plane monochromatic electromagnetic waves are commonly written in the form
\begin{equation} 
  \label{eq:polarization:e_field}
  \VctStl{E}(\VctStl{x},t) 
   = \left[E_v \atop E_h\right] e^{i(\VctStl{k}\VctStl{x} - \omega t)}
   = \left(E_v \eVrt +  E_h \eHor \right) 
      e^{i(\VctStl{k}\VctStl{x} - \omega t)}
\end{equation}
where $\VctStl{E}$ is the electric field vector, the subscripts $v$
and $h$ denote the components with vertical and horizontal
polarisation, respectively. $E_v$ and $E_h$, the amplitudes, are
complex numbers, $\VctStl{k}$ and $\omega$ are the wavenumber vector
and the angular frequency, respectively, of the plane wave, and the
unit vectors $\eVrt = (1,0)\Trp$, $\eHor = (0,1)\Trp$.  It is always
implicitly understood that the actual, physical, electric field is the
real part of the above expression. Rewriting the complex amplitudes
$E_v$ and $E_h$ using real, non-negative amplitudes $a_v$ and $a_h$, and
phases $\delta_v$ and $\delta_h$,
\begin{equation}
  \label{eq:polarization:compl_ampl}
  E_v=a_v e^{i\delta_v}\mbox{, }
  E_h=a_h e^{i\delta_h}
\end{equation}
the actual electric field vector $\VctStl{\tilde{E}}$ is
\begin{equation}
  \label{eq:polarization:actual_efield}
  \VctStl{\tilde{E}}(\VctStl{x},t) = \Re[\VctStl{E}(\VctStl{x},t)] 
    = \left[a_v\cdot \cos(\VctStl{k}\VctStl{x} - \omega t + \delta_v) 
                       \atop 
            a_h\cdot \cos(\VctStl{k}\VctStl{x} - \omega t + \delta_h) 
       \right] 
\end{equation}
In general, instruments do not measure the electric or magnetic field
vectors of an electromagnetic wave, but rather the time-averaged
intensity, i.e., the energy flux, $F$. This is the time-averaged Poynting
vector (which, in turn, is proportional to the square of the electric
field), thus:
\begin{eqnarray}
  F 
  &=& 
  \sqrt{\frac{\epsilon}{\mu}}\overline{(\VctStl{\tilde{E}}(\VctStl{x},t))^2}\\
   \nonumber
  &=&
  \sqrt{\frac{\epsilon}{\mu}}\left(
    a_v^2\overline{\cos^2(\VctStl{k}\VctStl{x} - \omega t + \delta_v)}
    + a_h^2\overline{\cos^2(\VctStl{k}\VctStl{x} - \omega t + \delta_h)}
   \right)
  \label{eq:polarization:intensity}
\end{eqnarray}
The overline denotes the time average
%, which is taken over one oscillation period $T$ ($T=2\pi\omega$). 
%Since the time average of
which for cosine squares is 1/2, thus:
\begin{equation}
  \label{eq:polarization:intensity2}
 F = 
  \half\sqrt{\frac{\epsilon}{\mu}}(
    a_v^2 + a_h^2)  
\end{equation}
Taking into account that for plane, monochromatic waves 
the time average always results in a factor
$\frac{1}{2}$, we can also directly write the intensity using the
electric field vector in complex notation
(Equation \ref{eq:polarization:e_field}).
\begin{eqnarray}
  \label{eq:polarization:intensity_from_complex}
  F &=&  \half\sqrt{\frac{\epsilon}{\mu}} \VctStl{E} (\VctStl{x},t) 
          \cdot \VctStl{E}^\ast (\VctStl{x},t)\\ \nonumber
    &=&   \half\sqrt{\frac{\epsilon}{\mu}}
          (E_v E_v^\ast + E_h E_h^\ast)
\end{eqnarray}
where the asterisk denotes complex conjugation.%
%\footnote{Note that the same argumentation works for ???}

%Commonly the factor  $\frac{1}{2}\sqrt{\frac{\epsilon}{\mu}}$ is
%omitted, and 
In addition to the flux, three more intensity quantities
are defined as in the following equations. They are called 
\emph{Stokes parameters}:
\begin{eqnarray}
  \label{eq:polarization:stokesparam_I}
  I &=& \half\InvImp ( E_v E_v^\ast + E_h E_h^\ast ) \\
  \label{eq:polarization:stokesparam_Q}
  Q &=&  \half\InvImp (  E_v E_v^\ast - E_h E_h^\ast ) \\
  \label{eq:polarization:stokesparam_U}
  U &=& - \half\InvImp (  E_v E_h^\ast + E_h E_v^\ast ) \\
  \label{eq:polarization:stokesparam_V}
  V &=& i \half\InvImp (E_h E_v^\ast - E_v E_h^\ast )
\end{eqnarray}
Written as a row or column vector, $(I,Q,U,V)$ is called
\emph{Stokes vector}. Note that sometimes, $S_0$, $S_1$, $S_2$, $S_3$
is used instead of $I$, $Q$, $U$, $V$.
Using the amplitude/phase notation from
Equation \ref{eq:polarization:compl_ampl},
we can rewrite the Stokes parameters as 
\begin{eqnarray}
  \label{eq:polarization:stokesparam_alt_I}
 I &=&  \half\InvImp (a_v^2 + a_h^2)\\
  \label{eq:polarization:stokesparam_alt_Q}
 Q &=& \half\InvImp (a_v^2 - a_h^2)\\
  \label{eq:polarization:stokesparam_alt_U}
 U &=&  - \InvImp a_v a_h \cos(\delta_v-\delta_h)\\
  \label{eq:polarization:stokesparam_alt_V}
 V &=&   - \InvImp  a_v a_h \sin(\delta_v-\delta_h)
\end{eqnarray}
The Stokes parameters fully characterise the electromagnetic wave and
therefore contain the same information as the electric field vector
(except for one absolute phase).  Since instruments generally measure
intensities (fluxes), describing electromagnetic radiation by the
Stokes parameters is more practical than describing it by the electric
(or magnetic) field vector. Furthermore, the Stokes parameters are
always real numbers.  Note that the Stokes parameters are sometimes
defined with different signs of $Q$, $U$, or $V$ (the definitions and
signs used here are based on
\citet{mishchenko00:_light_scatt_nonsp_partic}). Moreover, their
normalisation may vary. In particular, the Stokes parameters can be
normalised to represent radiance or irradiance (instead of intensity),
which is usually done in radiative transfer contexts.
 
In order understand what the Stokes parameters mean, we have to go
back to the electric field vector and see what polarisation state it
describes.  To do so, we look at the curve that the tip of the
physical electric field vector $\VctStl{\tilde{E}}$ describes with
time at a fixed position $\VctStl{x_0}$:
\begin{eqnarray}
  \tilde{E}_v (t) &=& a_v \cos(\Delta_v - \omega t)\\
  \tilde{E}_h (t) &=& a_h \cos(\Delta_h - \omega t)
\end{eqnarray}
where $\Delta_{v,h} = \VctStl{k}\VctStl{x_0} + \delta_{v,h}$. 
To see that this is an ellipse, we first split the cosines using
the addition theorem:
\begin{eqnarray}
  \label{eq:polarization:tip_of_fieldvec1}
  \tilde{E}_v (t) &=&   a_v \cos\Delta_v \cos(\omega t)
                      + a_v \sin\Delta_v \sin(\omega t)\\
  \label{eq:polarization:tip_of_fieldvec2}
  \tilde{E}_h (t) &=&   a_h \cos\Delta_h \cos(\omega t)
                      + a_h \sin\Delta_h \sin(\omega t)
\end{eqnarray}

In order to have the tip of $\VctStl{\tilde{E}}$ describe an ellipse with semi-major axis $a_0 \cos\beta$
and
semi-minor axis $a_0 \sin\beta$, where $a_0^2 = a_v^2 + a_h^2$, 
it should have the following form
\begin{eqnarray}
  \label{eq:polarization:ellipse_parallel}
 \tilde{E}_v (t) &=&   a_0 \sin\beta \cos(\omega t)\\
 \tilde{E}_h (t) &=&   a_0 \cos\beta \sin(\omega t)
\end{eqnarray}
Here $\beta$ must be between $-45\degree$ and $45\degree$: the tip of
the vector $\VctStl{\tilde{E}}$ describes a circle for $\beta = \pm
45\degree$ (circular polarisation), oscillates along the $h$-axis for
$\beta = 0$ (linear polarisation) and else describes an ellipse (cf.
Figure \ref{fig:polarization:ellipse_aligned}). The sense of rotation
is counterclockwise for positive $\beta$ (corresponding to left-circular
or left-elliptic polarisation) and clockwise for negative
$\beta$ (corresponding to right-circular
or right-elliptic polarisation).
%which corresponds to left-handed
Since $|\tan\beta|$ is the ratio of the semi-minor and semi-major axes
of the ellipse (the ellipticity), $\beta$ is called the ellipticity
angle.\label{def:ellipticity-angle}
\begin{figure}
 \begin{center}
  \begin{minipage}[c]{0.9\textwidth}
   \begin{center}
    \includegraphics*[width=0.9\hsize]{pol_ellipse_aligned}
   \end{center}
  \end{minipage}
  \begin{minipage}[c]{0.9\textwidth}
   \caption{The ellipse that the electric field vector describes with
     time, with the major axis oriented along the $h$-axis.}
   \label{fig:polarization:ellipse_aligned}
  \end{minipage}
 \end{center}
\end{figure}   
Note that the semi-major axis is oriented along the positive $h$-axis.
To have the major axis of the ellipse enclose an arbitrary angle
$\zeta$ ($0 \leq \zeta < 180\degree$) with the $h$-axis, we apply a
rotation matrix and get the equation for an ellipse with arbitrary
shape (ellipticity) and orientation (cf.\
Figure \ref{fig:polarization:ellipse_arbitrary})\label{def:orientation-angle}:
\begin{eqnarray}
  \label{eq:polarization:ellipse_rotated1}
 \tilde{E}_v (t) &=&  a_0(  \sin\beta \cos(\omega t) \cos\zeta
                           +\cos\beta \sin(\omega t) \sin\zeta )\\
  \label{eq:polarization:ellipse_rotated2}
 \tilde{E}_h (t) &=&  a_0( -\sin\beta \cos(\omega t) \sin\zeta
                           +\cos\beta \sin(\omega t) \cos\zeta )
\end{eqnarray}
%
\begin{figure}
 \begin{center}
  \begin{minipage}[c]{0.9\textwidth}
   \begin{center}
    \includegraphics*[width=0.9\hsize]{pol_ellipse_arbitrary}
   \end{center}
  \end{minipage}
  \begin{minipage}[c]{0.9\textwidth}
   \caption{The ellipse that the electric field vector describes with
     time, with the major axis oriented arbitrarily.}
   \label{fig:polarization:ellipse_arbitrary}
  \end{minipage}
 \end{center}
\end{figure}   
With these definitions, horizontal polarisation corresponds to
$\beta=0\degree$ and $\zeta=0\degree$; vertical polarisation to 
$\beta=0\degree$ and $\zeta=90\degree$; left-circular to 
$\beta=45\degree$ and any value of $\zeta$; right-circular to 
$\beta=-45\degree$ and any value of $\zeta$.
    
Now we want to establish a direct connection between the parameters
$\beta$ and $\zeta$ describing the shape (ellipticity) and orientation
of the polarisation ellipse on the one hand, and the amplitudes $a_v$
and $a_h$ and phases $\delta_v$ and $\delta_h$ of the components of
the electric field vector on the other hand.  Comparing the
$\sin(\omega t)$ and $\cos(\omega t)$ terms in
Equations \ref{eq:polarization:ellipse_rotated1} to
\ref{eq:polarization:ellipse_rotated2} with the corresponding terms
in Equations \ref{eq:polarization:tip_of_fieldvec1} to
\ref{eq:polarization:tip_of_fieldvec2}, we get:
\begin{eqnarray}
  \label{eq:polarization:corresp1a}
 a_v \cos\Delta_v &=& a_0 \sin\beta \cos\zeta\\
  \label{eq:polarization:corresp1b}
 a_v \sin\Delta_v &=& a_0 \cos\beta \sin\zeta
\end{eqnarray}
and 
\begin{eqnarray}
  \label{eq:polarization:corresp2a}
 a_h \cos\Delta_h &=& -a_0 \sin\beta \sin\zeta\\
  \label{eq:polarization:corresp2b}
 a_h \sin\Delta_h &=&  a_0 \cos\beta \cos\zeta
\end{eqnarray}
Multiplying Equation \ref{eq:polarization:corresp1a} with
Equation \ref{eq:polarization:corresp2a}, and
Equation \ref{eq:polarization:corresp1b} with
Equation \ref{eq:polarization:corresp2b} and adding up the results, we get
\begin{equation}
  a_v a_h (\cos\Delta_v\cos\Delta_h + \sin\Delta_v\sin\Delta_h)
  = a_0^2 \sin\zeta\cos\zeta (\cos^2\beta - \sin^2\beta) 
\end{equation}
Using the addition theorems for sinusoidals and taking into account
that
$\Delta_v-\Delta_h = \delta_v-\delta_h$:
\begin{equation}
  \label{eq:polarization:sinzetacosbeta}
  \frac{a_v a_h}{a_0^2} \cos(\delta_v-\delta_h)
  = \half\sin(2\zeta)\cos(2\beta)
\end{equation}
In a similar way, subtracting the product of
Equation \ref{eq:polarization:corresp1b} with
Equation \ref{eq:polarization:corresp2a} from the product of
Equation \ref{eq:polarization:corresp1a} with
Equation \ref{eq:polarization:corresp2b} and adding up the results, we get
\begin{equation}
  \label{eq:polarization:sinbeta}
  -\frac{a_v a_h}{a_0^2} \sin(\delta_v-\delta_h)
  = \half\sin(2\beta)
\end{equation}
The above two equations tell us how to translate the amplitudes
($a_v$, $a_h$) and phases ($\delta_v$, $\delta_h$) of the vertical and
horizontal component of the electric field into the orientation and
shape of the ellipse that the tip of the electric field vector describes with
time.  We can obtain one further relation by subtracting the sum of
the squares of Equation \ref{eq:polarization:corresp2a} and
Equation \ref{eq:polarization:corresp2b} from the sum of the squares of
Equation \ref{eq:polarization:corresp1a} and
Equation \ref{eq:polarization:corresp1b}:
\begin{equation}
  \label{eq:polarization:cosbetazeta}
 a_v^2 - a_h^2 =  -a_0^2 \cos(2\zeta)\cos(2\beta)
\end{equation}
Finally, we use the above 3 equations
(\ref{eq:polarization:sinzetacosbeta}, 
\ref{eq:polarization:sinbeta} and 
\ref{eq:polarization:cosbetazeta}) to rewrite the Stokes parameters
(Equations \ref{eq:polarization:stokesparam_alt_I} to
\ref{eq:polarization:stokesparam_alt_V}) 
as
\begin{eqnarray}
  \label{eq:polarization:stokesparam_alt2_I}
 I &=&  \half\InvImp a_0^2\\
  \label{eq:polarization:stokesparam_alt2_Q}
 Q &=&  - \half\InvImp a_0^2 \cos(2\zeta)\cos(2\beta)\\ 
  \label{eq:polarization:stokesparam_alt2_U}
 U &=& -\half\InvImp a_0^2 \sin(2\zeta)\cos(2\beta)\\
  \label{eq:polarization:stokesparam_alt2_V}
 V &=& -\half\InvImp a_0^2 \sin(2\beta)
\end{eqnarray}
FIXME: $\beta<0$ is right-handed pol. (see above, consistent with
Jackson and others); thus $V>0$. This conflicts with Mishchenko's book
(p.26).
  

Thus, we can get the orientation angle $\zeta$ of the ellipse from
\begin{equation}
  \label{eq:polarization:tan2zeta}
 \tan(2\zeta) = \frac{U}{Q}
\end{equation}
Since $0 \leq 2\zeta < 360\degree$, there are 2 solutions for $\zeta$ for a
given pair $U,Q$. This ambiguity is resolved by looking at
Equation \ref{eq:polarization:stokesparam_alt2_Q}, taking into account that
$|\beta| \leq 45\degree$ and thus $\cos(2\beta) \geq 0$:
The sign of $\cos(2\zeta)$ must be the same as the sign of $-Q$.

We get the ellipticity angle $\beta$  from
\begin{equation}
  \label{eq:polarization:tan2beta}
 \tan(2\beta) = - \frac{V}{(Q^2 + U^2)^{1/2}}  
\end{equation}

$I$ is the total intensity of the radiation, $Q$ is the difference in the
intensity of the vertically and horizontally polarised components (cf. Section
\ref{sec:polarization:measuring}). $I$ is always non-negative, and $Q$, $U$,
and $V$ are between $+I$ and $-I$, since they can be expressed as a product of
$I$ with sines and/or cosines (Equations
\ref{eq:polarization:stokesparam_alt2_Q} to
\ref{eq:polarization:stokesparam_alt2_V}). Note also that the 4 Stokes
parameters are not independent (for completely polarised radiation, see further
Section~\ref{sec:polarization:part_pol}), since the following equality applies:
\begin{equation}
  \label{eq:polarization:Isquare}
  I^2 = Q^2 + U^2 + V^2
\end{equation}
Some examples of Stokes parameters for specific polarisations are given
at the end of the next section (page \pageref{stokes-examples}).


\section{Measuring Stokes parameters}
%=====================================================================
\label{sec:polarization:measuring}
The three different ways given so far to write the Stokes parameters
(Equations \ref{eq:polarization:stokesparam_I}ff.,
Equations \ref{eq:polarization:stokesparam_alt_I}ff.,
Equations \ref{eq:polarization:stokesparam_alt2_I}ff.)  are not very
helpful if we actually want to measure the Stokes parameters. So here
we are going to rewrite them while keeping in mind that most
instruments can just measure intensities of radiation.

We have seen above that the Stokes parameter $Q$ is the difference in
the intensity of the vertically and horizontally polarised components 
(Equations \ref{eq:polarization:stokesparam_Q},
or \ref{eq:polarization:stokesparam_alt_Q})
\begin{equation}
  \label{eq:polarization:Q_Idiff}
  Q = I_v - I_h
\end{equation}
where
\begin{eqnarray}
  \label{eq:polarization:Iv}
  I_v &=&  \half\InvImp E_v E_v^\ast\\
  \label{eq:polarization:Ih}
  I_h &=&  \half\InvImp E_h E_h^\ast
\end{eqnarray}
  
Thus if we measure $I_v$ and $I_h$ using -- for optical wavelengths --
a polariser aligned with the $v$- and the $h$-axis, respectively, or
using -- for microwaves -- two appropriately aligned dipole antennas, we
can directly obtain $I$ by taking their sum and $Q$ by taking their
difference.

$U$ and $V$ can likewise be expressed as differences of intensities, but not
with respect to the linear base \eVrt and \eHor. We recall Equation
\ref{eq:polarization:e_field}, omitting the oscillatory term:
\begin{equation}
  \label{eq:polarization:E_linbase}
  \VctStl{E}= \left(E_v \eVrt +  E_h \eHor \right) 
\end{equation}

Now we want to write \VctStl{E} by two components along polarisation axes at
$\pm 45\degree$ with respect to the $h$-axes. The basis vectors are
thus (cf.\ Figure \ref{fig:polarization:e45})
\begin{eqnarray}
  \label{eq:polarization:e_p45}
  \ePls &=& \sqrt{\half} \left( \eHor - \eVrt \right) \\
  \label{eq:polarization:e_m45}
  \eMin &=& \sqrt{\half} \left( \eHor + \eVrt \right) 
\end{eqnarray}
and we get the field vector in this modified linear basis:
\begin{equation}
  \label{eq:polarization:E_lin45base}
  \VctStl{E} = \underbrace{
               \sqrt{\half}\left(E_v +  E_h \right)}_{E_{-45\degree}} 
               \eMin 
              + \underbrace{
               \sqrt{\half}\left( -E_v +  E_h \right)}_{E_{+45\degree}} 
               \ePls 
\end{equation}
%
\begin{figure}
 \begin{center}
  \begin{minipage}[c]{0.9\textwidth}
   \begin{center}
    \includegraphics*[width=0.4\hsize]{pol_e45}
   \end{center}
  \end{minipage}
  \begin{minipage}[c]{0.9\textwidth}
   \caption{Two sets of basis vectors for the linear basis.}
   \label{fig:polarization:e45}
  \end{minipage}
 \end{center}
\end{figure}   
With the definitions of intensities of the components,
\begin{eqnarray}
  \label{eq:polarization:Im45}
  I_{-45\degree} &=& \half\InvImp E_{-45\degree} E_{-45\degree}^\ast\\
  \label{eq:polarization:Ip45}
  I_{+45\degree} &=& \half\InvImp E_{+45\degree} E_{+45\degree}^\ast
\end{eqnarray}
we get for their difference:
\begin{eqnarray}
   I_{-45\degree} -  I_{+45\degree}
   &=&
     \half\InvImp\left[
      \half (E_v + E_h) (E_v^\ast + E_h^\ast)
       - \half (-E_v + E_h) (-E_v^\ast + E_h^\ast)\right]\\ \nonumber
   &=&
   \half\InvImp(E_v E_h^\ast + E_h E_v^\ast)
  \label{eq:polarization:Idiff_lin45base}
\end{eqnarray}
Therefore (cf. Equation \ref{eq:polarization:stokesparam_U})
\begin{equation}
  \label{eq:polarization:U_Idiff}
  U =   I_{+45\degree} -  I_{-45\degree}
\end{equation}
Thus if we measure $I_{+45\degree}$ and $I_{-45\degree}$ using -- for
optical wavelengths -- a polariser aligned at $+45\degree$ and
$-45\degree$ with respect to the $h$-axis, respectively, or using --
for microwaves -- two appropriately aligned dipole antennas, we can
directly obtain $U$ by taking their difference.

In order to see how to measure the fourth Stokes parameter, $V$, we
have to transform to the circular basis, i.e., express \VctStl{E} by a
left-hand ($LH$) and a right-hand ($RH$) circularly polarised component. The
relevant equations:

Basis vectors
\begin{eqnarray}
  \label{eq:polarization:e_Lh}
  \eLh &=& \sqrt{\half} \left( \eVrt + i\eHor \right) \\
  \label{eq:polarization:e_Rh}
  \eRh &=& \sqrt{\half} \left( \eVrt - i\eHor \right) 
\end{eqnarray}
%
%FIXME: \eRh{} and \eLh{} have been  swapped one...\\
Field vector in circular base
\begin{equation}
  \label{eq:polarization:E_circbase}
  \VctStl{E} = \underbrace{
               \sqrt{\half}\left(E_v -  i E_h \right)}_{E_{LH}} 
               \eLh 
              + \underbrace{
               \sqrt{\half}\left(E_v +  i E_h \right)}_{E_{RH}} 
               \eRh 
\end{equation}
%
Intensity of the components
\begin{eqnarray}
  \label{eq:polarization:ILH}
  I_{LH} &=& \half\InvImp E_{LH} E_{LH}^\ast\\
  \label{eq:polarization:IRH}
  I_{RH} &=& \half\InvImp E_{RH} E_{RH}^\ast
\end{eqnarray}
%
Their difference
\begin{eqnarray}
   I_{LH} -  I_{RH}
   &=&
     \half\InvImp \left[
     \half (E_v - i E_h) (E_v^\ast + i E_h^\ast)
    - \half (E_v + i E_h) (E_v^\ast - i E_h^\ast)\right]\\ \nonumber
   &=&
   i \half\InvImp (E_v E_h^\ast -  E_h E_v^\ast)
  \label{eq:polarization:Idiff_circbase}
\end{eqnarray}
Therefore (cf. Equation \ref{eq:polarization:stokesparam_V}):
\begin{equation}
  \label{eq:polarization:V_Idiff}
  V =   I_{RH} -  I_{LH}
\end{equation}
%
Thus if we measure $I_{RH}$ and $I_{LH}$ using -- for microwaves --
appropriate helical beam antennas, we can directly obtain $V$ by
taking their difference.  Unfortunately, for optical wavelengths, we
cannot measure $I_{RH}$ and $I_{LH}$ directly with the help of
filters like  polarisers and retarders%
\footnote{A retarder allows the phase of two orthogonal components 
of light to be varied  with respect to each other.}.  
However, a combination of a retarder and a polarizer can be used
to measure the sum of $I$ and $V$:

The light first passes through a retarder that delays the phase of
the horizontally polarised component by 90\degree{} with respect to
the phase of the vertically polarised component (a quarter-wave
plate).
A phase delay by 90\degree can be expressed as a multiplication of the
horizontal component by $i$, so the resulting
electric field vector $\VctStl{E}^\prime$ is
\begin{equation}
  \label{eq:polarization:E_retarded}
  \VctStl{E}^\prime = \left(E_v \eVrt +   i E_h \eHor \right) 
\end{equation}
The light then passes through a polarizer that is  aligned at
$-45\degree$ with respect to the $h$-axis. This means we have to
project $\VctStl{E}^\prime$ onto $\eMin$, resulting in
\begin{equation}
  \label{eq:polarization:E_retarded_polarized}
  \VctStl{E}^{\prime\prime} 
   = (\VctStl{E}^\prime \cdot \eMin)\eMin 
   = \sqrt{\half}\left(E_v  +  i E_h  \right) \eMin 
\end{equation}
Measuring the intensity now, we get
\begin{eqnarray}
  I^{\prime\prime} 
   &=&  \left| \VctStl{E}^{\prime\prime} \right| ^2 \\ \nonumber
   &=& \half\left(E_v +  i E_h  \right)  
     \left(E_v^\ast -  i E_h^\ast  \right)\\ \nonumber
   &=& \half \left( |E_v|^2 + |E_h|^2 - 
        i (E_v E_h^\ast -  E_h E_v^\ast )\right) \\ \nonumber
   &=& \half ( I + V )
  \label{eq:polarization:I_retarded_polarized}
\end{eqnarray}
Here is a summary of the Stokes parameters in terms of intensities of
orthogonal components:
\begin{eqnarray}
  \label{eq:polarization:stokesparam_Idiff_I}
  I &=& I_v + I_h = I_{-45\degree} + I_{+45\degree} = I_{RH} + I_{LH}\\
  \label{eq:polarization:stokesparam_Idiff_Q}
  Q &=&   I_v - I_h \\
  \label{eq:polarization:stokesparam_Idiff_U}
  U &=&  I_{+45\degree} - I_{-45\degree} \\
  \label{eq:polarization:stokesparam_Idiff_V}
  V &=&  I_{RH} - I_{LH}
\end{eqnarray}
We see that $Q$ and $U$ are both related to linear polarisation, while
$V$ is related to circular polarisation.

Here are the Stokes parameters for some standard polarisations:
\begin{center}
\label{stokes-examples}
\begin{tabular}{r@{\hspace{2em}(}r@{,}r@{,}r@{,}r@{)}}
\hline
polarisation & $I$ &   $Q$   &   $U$  &   $V$ \\ \hline
horizontal   & $I$ &  $-I$   &    0   &   0   \\ 
vertical     & $I$ &  $+I$   &    0   &   0   \\ 
linear $\pm 45\degree$
             & $I$ &   0     &$\mp I$ &   0   \\ 
right-circular& $I$ &  0     &    0   &   $I$ \\ 
left-circular& $I$ &   0     &    0   &  $-I$ \\ \hline
\end{tabular}
\end{center}



\section{Partial polarisation}
%=====================================================================
\label{sec:polarization:part_pol}
The equality   $I^2 = Q^2 + U^2 + V^2$
(Equation \ref{eq:polarization:Isquare}) 
%and in fact all of section \ref{sec:polarization:monochrom} 
is valid for the ideal case of a 
monochromatic plane wave that is completely polarised, i.e., where the
amplitudes $a_v$ and
$a_h$ and the phases $\delta_v$ and $\delta_v$ are fixed and do not
vary with time. This means that the plane wave is emitted by one
coherent source.

In reality, i.e., in the case of natural radiation, 
the amplitudes and phases fluctuate, since the radiation
originates from several sources that do not emit radiation coherently,
and since the emission from one source usually has very short
coherence times. This
means that we usually have a superposition of radiation from several incoherent
sources, and that the polarisation state of the radiation from each
source fluctuates as well\footnote{This does, of course, not apply to
  coherent sources like lasers or coherent radars.}. 
Typically, such fluctuations have time scales that are longer than the
period ($2\pi/\omega$) of the oscillation, but that are still shorter
than the integration time of the instrument that measures the
radiation. Thus, the instrument measures an incoherent
 superposition of time averages
over of the fluctuating polarisation. If the fluctuations are
random for all the sources and if the different sources emit 
incoherently and are not in any way oriented, then there is no preferred
orientation, ellipticity or handedness of the emitted radiation, which
is then called unpolarised. This is the case for radiation from the sun.
%, but also to
%a large extent for thermal radiation emitted by the atmosphere:
%While the emission from individual molecules might be polarized
%depending on the direction into which it is emitted, the random
%orientation of the molecules causes the resulting radiation to be
%unpolarized.
If the fluctuations are not completely random, the
radiation is called partially polarised.

To quantify this rather heuristic argumentation, we express the
above-mentioned ideas in the language of the Stokes parameters: The
Stokes parameters $I$, $Q$, $U$, $V$ derived from measurements result
from the superposition of radiation from many sources and/or the
average over emission events with individual Stokes parameters $I_i$,
$Q_i$, $U_i$, $V_i$.  Since the different sources and/or emission events are
incoherent, the Stokes parameters -- which are intensity, not
amplitude quantities -- can simply be added up:
\begin{equation}
  \label{eq:polarization:summed_stokes}
  I = \sum_i I_i \: \mbox{, }\; 
  Q = \sum_i Q_i \: \mbox{, }\; 
  U = \sum_i U_i \: \mbox{, }\; 
  V = \sum_i V_i
\end{equation}
In the case of unpolarised radiation, i.e., when the amplitudes and
phases, or equivalently, the orientation angle $\zeta$ and the
ellipticity angle $\beta$ are random (uniformly distributed), 
$Q$, $U$, and $V$
each cancel out.

The equality $I_i^2 = Q_i^2 + U_i^2 + V_i^2$
(cf. Equation \ref{eq:polarization:Isquare}) still holds for each
contribution $i$, but for the resulting $I,Q,U,V$, we have in general the
inequality
\begin{equation}
 \label{eq:polarization:stokes_inequality}
  I^2 \geq Q^2 + U^2 + V^2
\end{equation}
%Thus, we get for partially polarized radiation:
%% THIS IS WRONG!
%\begin{eqnarray}
%  I^2 &=& \left(\sum_i I_i\right)^2\\ \nonumber
%      &\geq& \sum_i I_i^2 = \sum_i (Q_i^2  + U_i^2 + V_i^2)\\ \nonumber
%      &=& Q^2 + U^2 + V^2
%  \label{eq:polarization:stokes_inequality}
%\end{eqnarray}
%The inequality follows from  the mathematical fact that the square of
%a sum of non-negative numbers is greater or equal than the sum of the
%squares. 
To prove it, we must once again go back to the amplitude/phase
notation (Equations \ref{eq:polarization:stokesparam_alt_I}ff.), also
cf. \citet[][chap. I.15]{chandrasekhar:60}, but we shall omit the
factor \half\InvImp{} on the right-hand sides, for the sake of better readability:
%, with the additional
%superscript $(i)$ for the components to be added:
\begin{eqnarray}
I &=& \sum_i I_i = \sum_i \left(a_v^{(i)}\right)^2
                    + \sum_i \left(a_h^{(i)}\right)^2 \\
Q &=& \sum_i Q_i = \sum_i \left(a_v^{(i)}\right)^2
                    - \sum_i \left(a_h^{(i)}\right)^2 \\
U &=& \sum_i U_i = - 2 \sum_i a_v^{(i)} a_h^{(i)} \cos\delta^{(i)} \\
 V &=& \sum_i V_i = 2 \sum_i a_v^{(i)} a_h^{(i)} \sin\delta^{(i)} \\
\end{eqnarray}
where $\delta^{(i)} = \delta_v^{(i)} - \delta_h^{(i)}$. 
We get
\begin{eqnarray}
\label{eq:polarization:square-difference}  
I^2 - Q^2 - U^2 - V^2 &=& 
  4 \sum_i \left(a_v^{(i)}\right)^2
                     \sum_i \left(a_h^{(i)}\right)^2 \\ \nonumber
  &&- 4 \left( \sum_i a_v^{(i)} a_h^{(i)} \cos\delta^{(i)} \right)^2
  \\ \nonumber
  &&- 4 \left( \sum_i a_v^{(i)} a_h^{(i)} \sin\delta^{(i)} \right)^2 
\end{eqnarray}
The first term on the right-hand side can be rearranged as
\begin{equation}
  \sum_i \left(a_v^{(i)}a_h^{(i)}\right)^2
  + \sum_{{i,j}\atop{i\not=j}} \left(a_v^{(i)}a_h^{(j)}\right)^2
\end{equation}
The other two terms can be rearranged similarly to yield:
\begin{eqnarray}
 &&- \sum_i \left( a_v^{(i)} a_h^{(i)} \right)^2 \
           \left[\cos^2\delta^{(i)} + \sin^2\delta^{(i)} \right]\\ \nonumber
 &&- \sum_{{i,j}\atop{i\not=j}} a_v^{(i)} a_h^{(i)} a_v^{(j)} a_h^{(j)} 
    \left[\cos\delta^{(i)}\cos\delta^{(j)}  
          + \sin\delta^{(i)}\sin\delta^{(j)} \right] 
\end{eqnarray}
Putting this into Equation \ref{eq:polarization:square-difference} (and
dividing by 4), the sums over just $i$ cancel and we get:
\begin{eqnarray}
\label{eq:polarization:square-difference2}  
(I^2 - Q^2 - U^2 - V^2)/4 &=&
  \sum_{{i,j}\atop{i\not=j}} \left(a_v^{(i)}a_h^{(j)}\right)^2\\ \nonumber
 &&- \sum_{{i,j}\atop{i\not=j}} a_v^{(i)} a_h^{(i)} a_v^{(j)} a_h^{(j)} 
    \cos(\delta^{(i)}-\delta^{(j)}) 
\end{eqnarray}
where the cosine addition theorem was used.  In the summation, we now
change from $i \not= j$ to $i<j$, so we have to symmetrise the first
term (the second term is already symmetric with respect to $i$ and $j$ and
therefore just gets a factor 2):
\begin{eqnarray}
\label{eq:polarization:square-difference3}  
(I^2 - Q^2 - U^2 - V^2)/4 &=&
  \sum_{{i,j}\atop{i<j}}\left[ 
   \left(a_v^{(i)}a_h^{(j)}\right)^2
   +\left(a_v^{(j)}a_h^{(i)}\right)^2 \right. \\ \nonumber
 &&\qquad\left.
 - 2 \left(a_v^{(i)} a_h^{(j)} \right) \left(a_v^{(j)} a_h^{(i)}\right) 
    \cos(\delta^{(i)}-\delta^{(j)}) \right]
\end{eqnarray}
Each summand of the sum on the right-hand side is positive, since it
is greater than or equal to $(a_v^{(i)}a_h^{(j)} -
a_v^{(j)}a_h^{(i)})^2$, which completes the proof.
The right-hand side vanishes only if $\delta^{(i)}=\delta^{(j)}$ and
$a_v^{(i)}/a_h^{(i)} = a_v^{(j)}/a_h^{(j)}$ for all $i,j$, i.e., if
the phase difference and amplitude ratio between the horizontal and
vertical component of the electric field is the same for all
contributions, in other words: if all contributions have the same
polarisation.

For completeness, we shall now restate the definition of the Stokes
component, extended to include natural radiation (i.e., including the
case of partially polarised and unpolarised radiation).
Instead of summing over the individual emission events, we use
ensemble averages, denoted by angular brackets:
\begin{eqnarray}
  \label{eq:polarization:stokesparam_I_general}
  I &=& \half\InvImp  \EnsAvr{E_v E_v^\ast + E_h E_h^\ast} \\
  \label{eq:polarization:stokesparam_Q_general}
  Q &=&  \half\InvImp \EnsAvr{  E_v E_v^\ast - E_h E_h^\ast} \\
  \label{eq:polarization:stokesparam_U_general}
  U &=& - \half\InvImp \EnsAvr{  E_v E_h^\ast + E_h E_v^\ast} \\ % 2012-08-30 Changed from E_v E_h^\ast - E_h E_v^\ast, revert if wrong //Richard Larsson
  \label{eq:polarization:stokesparam_V_general}
  V &=& i \half\InvImp \EnsAvr{E_h E_v^\ast - E_v E_h^\ast}
\end{eqnarray}
Except for the ensemble average \EnsAvr{..}, the definition is identical to the
one for monochromatic, plane waves (Equations
\ref{eq:polarization:stokesparam_I} to \ref{eq:polarization:stokesparam_V}).
The same applies to the second and third definitions of the Stokes parameters
(Equations \ref{eq:polarization:stokesparam_alt_I} to
\ref{eq:polarization:stokesparam_alt_V} and Equations
\ref{eq:polarization:stokesparam_alt2_I} to
\ref{eq:polarization:stokesparam_alt2_V}, respectively). Note that the fourth
definition (Equations \ref{eq:polarization:stokesparam_Idiff_I} to
\ref{eq:polarization:stokesparam_Idiff_V}) which uses sums and differences of
intensities, is equally valid for fully polarised, partially polarised and
unpolarised radiation. The definition of intensities, however, has to include
the ensemble average: $I_h = \EnsAvr{E_h E_h^\ast}$ etc.

Now we can define a measure for
the degree of polarisation, $p$, as:
\begin{equation}
  \label{eq:polarization:pol_degree}
  p = \frac{\sqrt{Q^2 + U^2 + V^2}}{I}
\end{equation}
For completely polarised radiation, $Q^2 + U^2 + V^2 = I^2$, so $p =
1$, and for unpolarised radiation, $Q = U = V = 0$, so $p = 0$.

Furthermore, it can be convenient to define the polarised
component of radiation by
\begin{equation}
  \label{eq:polarization:pol_component}
  I_p^2 = Q^2 + U^2 + V^2
\end{equation}
and the unpolarised component as
\begin{equation}
  \label{eq:polarization:unpol_component}
  I_u = I - I_p
\end{equation}
Thus, partially polarised radiation, described by a Stokes vector $(I,
Q, U, V)$, can be regarded as as a superposition of completely
polarised radiation described by the Stokes vector $(I_p, Q, U, V)$
and unpolarised radiation described by the Stokes vector $(I_u,
0,0,0)$.  We see that the Stokes parameter formalism can conveniently
deal with partially polarised and with unpolarised radiation, much in
contrast to the formalism using the electric field (amplitude and
phase).

In addition to the degree of polarisation, $p$, we can define measures
for the circularity and the linearity of the polarisation.
Recalling Equations \ref{eq:polarization:stokesparam_Idiff_U}
and \ref{eq:polarization:stokesparam_Idiff_V},
we can define the degree of linear polarisation, $p_{lin}$, as
\begin{equation}
  \label{eq:polarization:p_lin}
 p_{lin} = \frac{\sqrt{Q^2 + U^2}}{I} 
\end{equation}
and the the degree of circular polarisation, $p_{circ}$, as
\begin{equation}
  \label{eq:polarization:p_circ}
 p_{circ} = \frac{V}{I} 
\end{equation}
%(Stephens p.61)


\subsection{Polarisation of Radiation in the Atmosphere}
\label{sec:polarization:atmosphere}
The radiation encountered in atmospheric sounding (for which ARTS is
intended) is natural radiation, coming from the sun, space (cosmic
background), and/or the atmosphere and the Earth surface (thermal
radiation, scattered radiation)\footnote{This is not so for active
  sounding techniques that use a coherent source, such as lidar.}.
Radiation from the sun is unpolarised, as already mentioned; the same
applies for the cosmic background.  In contrast,
radiation emitted by the ground can be polarised, dependent on
material, texture and direction.  Radiation emitted by the atmosphere
(thermal radiation) is almost unpolarised because of the random
orientation of the air molecules.  An exception is caused by the
Zeeman effect induced in oxygen molecules by the -- anisotropic --
Earth's magnetic field.  Scattering of radiation by oriented
particles, e.g.\ cirrus clouds, is sensitive to polarisation, and
generally increases the degree of polarisation.  Typically $I > |Q| >
|U|,|V|$.


\subsection{Antenna polarisation}
\label{sec:polarization:antenna}
Finally we want to know what an antenna of arbitrary polarisation
response (antenna polarisation)
measures if radiation of some other arbitrary polarisation is incident
on it.

In order to clarify the concept, we first consider some trivial examples:
We assume an antenna that receives only vertically polarised
radiation.
\begin{itemize}
\item If the incident radiation is fully horizontally polarised, the
  antenna will measure nothing.
\item If the incident radiation is fully vertically polarised, the
  antenna will measure the full intensity of the radiation.
\item If the radiation is fully left- or right-circularly polarised,
  the antenna will measure half of the full intensity, for circularly
  polarised radiation is made up of equal portions of vertically and
  horizontally polarised radiation, superimposed with a phase lag of
  90\degree.
\end{itemize}

In order to be able to describe the general case, we first have to
formalise the description of the antenna polarisation.  Polarised
radiation is described by
\begin{enumerate}
\item 
the Jones vector, or 
\item the Stokes
vector, or
\item intensity, $I$, orientation angle, $\zeta$
(i.e., the angle between the major axis of the polarisation ellipse and
the horizontal polarisation direction), %see \pageref{def:orientation-angle}), 
and ellipticity angle, $\beta$ (see
page \pageref{def:ellipticity-angle}).
\end{enumerate}
Since the intensity of the
radiation is the absolute square (the squared ``length'') of the
complex Jones vector, or, in other words, the first Stokes component,
$I$, the polarisation alone is defined by
\begin{enumerate}
\item a normalised Jones
vector, or
\item three normalised Stokes components $Q$, $U$, and $V$
(where $Q^2 + U^2 + V^2 = 1$), or 
\item the orientation angle $\zeta$
and the ellipticity angle $\beta$ (see
Equation \ref{eq:polarization:tan2zeta} to \ref{eq:polarization:tan2beta}).
\end{enumerate}
In the same way, the polarisation of the antenna can be described in
one of three ways:
\begin{enumerate}
\item a normalised Jones vector
\begin{equation}
\VctStl{e}= \left[e_v \atop e_h\right]\qquad
   \mbox{where}\quad \VctStl{e}\cdot\VctStl{e}^\ast = 1
 \end{equation}
(note  that in the scalar product of two complex vectors, the second one
has to be complex-conjugated.)

\item a normalised Stokes vector 
\begin{equation}
\VctStl{i} = (1, q, u, v)\qquad
   \mbox{where}\quad q^2+u^2+v^2=1
 \end{equation}
 
\item the
  two angles $\zeta$ and $\beta$.  According to
  Equation \ref{eq:polarization:stokesparam_alt2_I}
  to \ref{eq:polarization:stokesparam_alt2_V}, we have:
\begin{eqnarray}
  \label{eq:polarization:antenna-q}
 q &=&  - \cos(2\zeta)\cos(2\beta)\\ 
  \label{eq:polarization:antenna-u}
 u &=& - \sin(2\zeta)\cos(2\beta)\\
  \label{eq:polarization:antenna-v}
 v &=& - \sin(2\beta)
\end{eqnarray}
\end{enumerate}
Now we can calculate the intensity $I^\prime$ 
the antenna measures. In terms of the electrical fields, 
i.e., Jones vectors,  we just have to
project the Jones vector \VctStl{E} of the incident radiation onto the
normalised Jones vector \VctStl{e} of the antenna,
\begin{equation}
  \label{eq:polarization:antenna-projection}
   \VctStl{E}^\prime =  (\VctStl{E}\cdot\VctStl{e}^\ast) \VctStl{e}
\end{equation}
(this is in effect like passing through a polarizer) and then take its
absolute square
\begin{equation}
  \label{eq:polarization:antenna:intensity-jones}
  I^\prime =  \half\InvImp\left| \VctStl{E}^{\prime} \right| ^2 
   = \half\InvImp\left|(\VctStl{E}\cdot\VctStl{e}^\ast) \right| ^2 
\end{equation}
With some elementary algebra (mainly using that $\half\InvImp E_v
E_v^\ast = (I+Q)/2$, $\half\InvImp E_h E_h^\ast = (I-Q)/2$,
$\half\InvImp E_v E_h^\ast = -(U - \mi V)/2$ which follow immediately
from Equation \ref{eq:polarization:stokesparam_I}
to \ref{eq:polarization:stokesparam_V} ) this can be rewritten in
terms of the of the Stokes vector \VctStl{I} of the incident radiation
and the Stokes vector \VctStl{i} of the antenna.  It turns out to be
just a scalar product:
\begin{equation}
  \label{eq:polarization:antenna:intensity-stokes}
  I^\prime = \frac{1}{2}\VctStl{i}\cdot\VctStl{I}
\end{equation}

%This is $S_{\mathrm{H_2 O}}$.

%This is $S_{H_2 O}$.

%This is $S_{O2}$.



\section{The scattering amplitude matrix}
\label{sec:polarization:ampmatrix}
%=====================================================================
The electric field, $[E_v,E_h]^T$, originating from a single scattering event
of an incident electric field $[E_v^0,E_h^0]^T$ may in the far field be written
as (see further Equation \ref{eq:rtetheory:amplitude_matrix})
\begin{equation}
  \label{eq:polarisation:ampmatrix1}
   \left[ \begin{array}{c} E_v \\ E_h \end{array} \right] =
   f(r)
   \left[ \begin{array}{cc} S_2 & S_3 \\ S_4 & S_1 \end{array} \right]
   \left[ \begin{array}{c} E_v^0 \\ E_h^0 \end{array} \right], 
\end{equation}
where $S_j$ are the scattering amplitude functions and all distance
effects are put into the function $f(r)$. Using Stokes based
nomenclature, the equation above becomes
\begin{equation}
  \label{eq:polarisation:ampmatrix2}
   \left[ \begin{array}{c} I\\Q\\U\\V \end{array} \right] =
   g(r)\MtrStl{F}
   \left[ \begin{array}{c} I^0\\Q^0\\U^0\\V^0 \end{array} \right],
\end{equation}
where all distance effects are put into the function $g(r)$ and the
transformation matrix \MtrStl{F} can be expressed as \citep[Sec.\ 5.4.3]{liou:02}.
\begin{equation}
  \label{eq:polarisation:Fmatrix}
  \MtrStl{F} = \left[ \begin{array}{cccc} 
    \scriptstyle \frac{1}{2}(M_2+M_3+M_4+M_1) & 
    \scriptstyle \frac{1}{2}(M_2-M_3+M_4-M_1) & 
    \scriptstyle S_{23}+S_{41} & \scriptstyle -D_{23}-D_{41} \\
    \scriptstyle \frac{1}{2}(M_2+M_3-M_4-M_1) & 
    \scriptstyle \frac{1}{2}(M_2-M_3-M_4+M_1) & 
    \scriptstyle S_{23}-S_{41} & \scriptstyle -D_{23}+D_{41} \\
    \scriptstyle S_{24}+S_{31} & \scriptstyle S_{24}-S_{31} & 
    \scriptstyle S_{21}+S_{34} & \scriptstyle -D_{21}+D_{34} \\
    \scriptstyle D_{24}+D_{31} & \scriptstyle D_{24}-D_{31} & 
    \scriptstyle D_{21}+D_{34} & \scriptstyle S_{21}-S_{34} 
  \end{array} \right].
\end{equation}
The elements of \MtrStl{F} are finally given by the following
expressions:
\begin{eqnarray}
  M_k &=& |S_k|^2, \\
  S_{kj} = S_{jk} &=& (S_jS_k^\ast+S_kS_j^\ast)/2, \\
  -D_{kj} = D_{jk} &=& i(S_jS_k^\ast-S_kS_j^\ast)/2, \qquad j,k= 1, 2, 3, 4.
\end{eqnarray}
Depending on the properties of the scattering event, the structure of
the matrix \MtrStl{F} differs. Two special cases are:
\begin{eqnarray}
  \label{eq:Fcase1}
  S_1=S_2,\quad S_3=S_4=0 &\to&    \MtrStl{F} =
      \left[\begin{array}{cccc}
        x&0&0&0\\
        0&x&0&0\\
        0&0&x&0\\
        0&0&0&x
      \end{array}
      \right], \\
  \label{eq:Fcase2}
  S_3=S_4=0 &\to&    \MtrStl{F} =
      \left[\begin{array}{cccc}
        x&x&0&0\\
        x&x&0&0\\
        0&0&x&x\\
        0&0&x&x
      \end{array}
      \right],
\end{eqnarray}
where $x$ indicates elements deviating from 0. Many (most?)  natural
materials have the property that $S_4$ is the complex conjugate of
$S_3$ $(S_3=S_4^\ast)$ and this results in that \MtrStl{F} is a
symmetric matrix (in general with all element positions filled).



%%% Local Variables: 
%%% mode: latex
%%% TeX-master: "uguide"
%%% End: 
% LocalWords:  ext matrix abs vec pha pnd sca lat lon za aa pt FIXME Eq LH RH
% LocalWords:  Eqs mishchenko scatt nonsp partic RTE Melsheimer overline Zeeman
% LocalWords:  Tsang
