\chapter{Integration functions}
%--------------------------------------------------------------------------
\label{sec:integration}

\starthistory
  220802 & Created and written by Sreerekha T.R.\\
  220103 & Included mathematical description for implemented integration method(CE).\\
\stophistory

%
% Introduction
%
A radiative transfer model which takes into account the effect of
scattering involves integration of certain quantities over the angles
of observation.  For example, from Section
\ref{U-sec:scattering:VRTE_sol} in \user\ it is clear that computing
scattering cross-section and scattering integral term requires
integration over zenith and azimuth directions. There are a wide range
of methods that can be used for numerical integration. They can be
used depending on various factors starting from how accurate the
result should be to the behaviour of the function. The one which is
implemented in ARTS is the trapezoidal integration method.


\section{Implementation files}
%-------------------------------------------------------------------------
\label{sec:integration:files}

The integration functions can be found in the files:
\begin{itemize}
\item \fileindex{math\_funcs.h}
\item \fileindex{math\_funcs.cc}
\end{itemize}
The implementation function \shortcode {AngIntegrate\_trapezoid}is
discussed in the second file. 

\section{Trapezoidal Integration}
%------------------------------------------------------------------------
\label{sec:integration:trapezoidal}

Trapezoidal Integration method comes under the Newton-Cotes formulas
where integration of a function is approximated by the area under the
curve described by the function.  Trapezoidal integration assumes that
the area under the curve is trapezoid.  

Trapezoidal rule : 
\begin{eqnarray}
\label{eq:trapezoidal_rule}
{\int_{x_1}^{x_2} f(x)dx}  = \frac{1}{2} h (f_1 + f_2) + O(h^3 f^{''})
\end{eqnarray}
This is a two-point formula ($x_1$ and $x_2$).  It is exact for
polynomials upto and including degree 1, i.e., f(x) = x. $O(h^3
f^{''})$ signifies how far is the true answer from the estimate. 

If we use eq. \ref{eq:trapezoidal_rule} $N - 1$ times, to do the
integration in the intervals $(x_1, x_2)$,  $(x_2, x_3)$, ...,
$(x_{N-1}, x_N)$, and then add the results, we obtain extended formula
for the integral from $x_1$ to $x_N$.

Extended Trapezoidal rule :
\begin{eqnarray}
\label{eq:ext_trapezoidal_rule}
{\int_{x_1}^{x_N} f(x)dx}  = \frac{1}{2} h \left [f_1 + 2(f_2 + f_3 +
... +f_{N-1})+f_N \right] + O\left [ \frac {(b-a)^3 f^{''}}{N^2} \right]
\end{eqnarray}

The last term tells how much the error will be decreased by taking
more number of steps. 

\section{Solid Angle Integration}
%------------------------------------------------------------------------
\label{sec:integration:solid_angle}
In our scattering problem, we are often encountered with a double integration
of functions over zenith and azimuth angles (see \user, Chapter
\ref{U-sec:scattering}).  One way to achieve
double integration is to use repeated
one-dimensional trapezoidal integration.  This is effective of course
only if the boundary is simple and the function is very smooth.  If
the function is strongly peaked and if know where it occurs, integral
should be broken into smaller regions so that the 
integrand is smooth in each.  Another thing is to take into account
the symmetry of the function as well as the boundary. For example in
our case, if the radiation is symmetric about the azimuth, the
integration in that direction returns constant value of $2 \pi$ and we
need to do only integration over zenith directions.  

The general form of a solid angle integration is
\begin{equation}
  \label{eq:solid_int}
S = \int_{4\pi} f(\omega) \DiffD\omega
\end{equation}
In spherical coordinates we can write:
\begin{equation}
  \label{eq:sol_int_sph}
 S = \int_0^\pi \int_0^{2\pi} f(\theta,\phi) \sin\theta \quad\DiffD\theta\DiffD\phi
\end{equation}
A double integration can be splitted into two single integrations:
\begin{eqnarray}
 S &=& \int_0^\pi \left(\int_0^{2\pi}  f(\theta,\phi) \sin\theta \DiffD\phi \right) \DiffD\theta \\
 &=& \int_0^\pi g(\theta) \DiffD\theta
\end{eqnarray}
If we have to integrate a vector, we can apply this method componentwise.

To solve the integral numerically we discretize $\theta$ and $\phi$ and obtain two angular grids ( $[\theta_0, \theta_1, \cdots, \theta_n]$  and $[\phi_0, \phi_1, \cdots, \phi_m]$). 
Then we can first calculate $g(\theta_j)$ for all $\theta_j$ unsing the trapezoidal method.
\begin{equation}
  g(\theta_j) = \sum_{i=1}^m \sin\theta_j \frac{f(\theta_j, \phi_i) + f(\theta_j, \phi_{i+1})}{2} \cdot (\phi_{i+1} - \phi_i)  
\end{equation}
The final step is to sum up all $g(\theta_j)$, again applying the trapezoidal method.
\begin{equation}
  S = \sum_{j=1}^n \frac{g(\theta_j) + g(\theta_{j+1})}{2} \cdot  (\theta_{j+1} - \theta_j)  
\end{equation}

If the radiation is symmetric about the azimuth we just calculate:
\begin{equation}
  S_{sym} = 2\pi \int_0^{\pi} f(\theta) \sin(\theta) \DiffD \theta 
\end{equation}
Unsing the trapezoidal method this can be written as:
\begin{equation}
  S_{sym} =  2\pi \sum_{j=1}^n \frac{h(\theta_j) + h(\theta_{j+1})}{2} \cdot  (\theta_{j+1} - \theta_j)  
\end{equation}
where $h(\theta) = \sin\theta\cdot f(\theta)$.
 
\vspace{2ex}



The function  \shortcode{AngIntegrate\_trapezoid} takes as input the integrand and the angles over which
the integration has to be done. For example in this case it can be the
zenith and azimuth angle
grid.
\begin{code}  
Numeric AngIntegrate_trapezoid(MatrixView Integrand,
                               ConstVectorView za_grid,
                               ConstVectorView aa_grid)
\end{code}
The integrand has the same number of rows as zenith angle grid
and columns as azimuth angle grid.  The inner loop does trapezoidal
integration of the integrand over all azimuth angles and the result is
stored in a Vector  res1[i]. Note that the integrand at every point
has to be multiplied with \shortcode {sin (za\_grid[i] * DEG2RAD)}
since we are integrating over solid angles.  The outer loop 
does an integration of res1[i] over all zentih angles.  The result of
this is returned back to the calling function.  


%%% Local Variables: 
%%% mode: latex
%%% TeX-master: t
%%% End: 
